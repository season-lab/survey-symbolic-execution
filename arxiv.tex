% v2-acmsmall-sample.tex, dated March 6 2012
% This is a sample file for ACM small trim journals
%
% Compilation using 'acmsmall.cls' - version 1.3 (March 2012), Aptara Inc.
% (c) 2010 Association for Computing Machinery (ACM)
%
% Questions/Suggestions/Feedback should be addressed to => "acmtexsupport@aptaracorp.com".
% Users can also go through the FAQs available on the journal's submission webpage.
%
% Steps to compile: latex, bibtex, latex latex
%
% For tracking purposes => this is v1.3 - March 2012

\documentclass[10pt,a4paper]{article} % Aptara syntax

\usepackage{authblk}
\usepackage{breakcites}

% Package to generate and customize Algorithm as per ACM style
\usepackage[ruled]{algorithm2e} 
\renewcommand{\algorithmcfname}{ALGORITHM}
\SetAlFnt{\small}
\SetAlCapFnt{\small}
\SetAlCapNameFnt{\small}
\SetAlCapHSkip{0pt}
\IncMargin{-\parindent}

\newcommand{\fullver}{}
\newcommand{\arxivver}{}
% !TEX root = main.tex
%\usepackage{a4wide}
\usepackage{listings}
\usepackage{comment}
\usepackage{amsmath}
\usepackage{graphicx}
\usepackage{amssymb}
\usepackage{url}
\usepackage{hyperref}
\usepackage{float}
\usepackage{lipsum}
\usepackage{caption}
\usepackage{subcaption}
\usepackage{adjustbox}
\usepackage{framed}
\usepackage{multirow}
\usepackage{framed}
\usepackage{enumitem}
\usepackage{epigraph}
\usepackage{wasysym} % \brokenvert
\usepackage{wrapfig}

%\usepackage[usenames, dvipsnames]{xcolor} % for acmart

% commands
%\newcommand{\fullver}{}
\ifdefined\fullver
\newcommand{\iffullver}[2]{#1}
\else
\newcommand{\iffullver}[2]{#2}
\fi

%\usepackage{tikz}
%\newcommand*\circled[1]{\tikz[baseline=(char.base)]{
%            \node[shape=circle,draw,inner sep=2pt] (char) {#1};}}


\renewcommand{\epigraphsize}{\footnotesize}
\setlength{\epigraphwidth}{10cm}
%\renewcommand{\epigraphrule}{0pt}

\definecolor{shadecolor}{rgb}{0.92,0.92,0.92}

\hypersetup{
  colorlinks = true, % colours links instead of ugly boxes
  urlcolor = blue, %  colour for external hyperlinks
  linkcolor = black, % colour of internal links
  citecolor = black, % colour of citations
  pdftitle = {A Survey of Symbolic Execution Techniques},
  pdfauthor= {Roberto Baldoni, Emilio Coppa, Daniele Cono D'Elia, Camil Demetrescu, Irene Finocchi}
}	


\newcommand{\mytempedit}[1]{\ignorespaces#1}
\newcommand{\revedit}[1]{{\color{blue}#1}}


\setlength{\FrameSep}{2pt}
\newcommand{\myparagraph}[1]{\medskip\noindent{\bf\small #1.} }
\newcommand{\myparagraphnoperiod}[1]{\medskip\noindent{\bf\small #1} }

% EDIT TO ENABLE NOTES
\newcommand{\mynote}[1]{\ignorespaces} % TODO
%\newcommand{\mynote}[1]{\marginpar{\raggedleft{\fontfamily{pbk}\selectfont\scriptsize{\em #1}}}}

\newcommand{\stwoe}{\text{S\textsuperscript{2}E}}
\newcommand{\myinput}[1]{\ifdefined\internalrep \input{../#1} \else \input{#1} \fi}
%\newcommand{\boxedexample}[1]{\vspace{2mm}\noindent\fbox{\parbox{0.98\textwidth}{{\em Example.} #1}}}

\ifdefined\arxivver
\newcommand{\boxedexample}[1]{
\begin{shaded}
\noindent{\bf\small Example.} #1
\end{shaded}
}
\else
\newcommand{\boxedexample}[1]{
%\vspace{-2mm}
\begin{shaded*}
\noindent{\bf\small Example.} #1
\end{shaded*}
%\vspace{-2mm}
}
\fi




% Document starts
\begin{document}

% Page heads
\markboth{R. Baldoni, E. Coppa, D. C. D'Elia, C. Demetrescu, and I. Finocchi}{A Survey of Symbolic Execution Techniques}

% Title portion
\title{\fontsize{22}{12}\selectfont{A Survey of Symbolic Execution Techniques}}
\author[1]{Roberto Baldoni}
\author[2]{Emilio Coppa}
\author[2]{Daniele Cono D'Elia}
\author[2]{\authorcr Camil Demetrescu}
\author[2]{Irene Finocchi}
\affil[1]{\small\href{http://www.cis.uniroma1.it/}{Cyber Intelligence and Information Security Research Center}, Sapienza University of Rome}
\affil[2]{\href{season-lab.github.io}{SEASON Lab}, Sapienza University of Rome}
\affil[ ]{{\vskip 1pt}\textit {\{baldoni,coppa,delia,demetres\}@dis.uniroma1.it\\ finocchi@di.uniroma1.it}}

\date{\vspace{-4mm}}

\maketitle

\begin{abstract}
Many security and software testing applications require checking whether certain properties of a program hold for any possible usage scenario. For instance, a tool for identifying software vulnerabilities may need to rule out the existence of any backdoor to bypass a program's authentication. One approach would be to test the program using different, possibly random inputs. As the backdoor may only be hit for very specific program workloads, automated exploration of the space of possible inputs is of the essence. Symbolic execution provides an elegant solution to the problem, by systematically exploring many possible execution paths at the same time without necessarily requiring concrete inputs. Rather than taking on fully specified input values, the technique abstractly represents them as symbols, resorting to constraint solvers to construct actual instances that would cause property violations. Symbolic execution has been incubated in dozens of tools developed over the last four decades, leading to major practical breakthroughs in a number of prominent software reliability applications. The goal of this survey is to provide an overview of the main ideas, challenges, and solutions developed in the area, distilling them for a broad audience.
\end{abstract}

% We no longer use \terms command
%\terms{Design, Algorithms, Performance}

%\keywords{Symbolic execution, static analysis, concolic execution, malware analysis}

%\acmformat{Roberto Baldoni, Emilio Coppa, Daniele Cono D'Elia, Camil Demetrescu, and Irene Finocchi, 2016. A survey of symbolic execution techniques.}

\iffalse
\begin{bottomstuff}
Author's addresses: R. Baldoni, E. Coppa, D.C. D'Elia, and C. Demetrescu, Department of Computer, Control, and Management Engineering, Sapienza University of Rome; I. Finocchi, Department of Computer Science, Sapienza University of Rome. 
\end{bottomstuff}
\fi



% % !TEX root = CSUR/main.tex

\section{Introduction}

Symbolic execution is a powerful program analysis technique introduced in the mid 70's in the context of software testing (see, e.g.,~{\cite{K-CACM76} and~\cite{H-TSE77}})\mynote{IF: are references appropriate? Should we give more credits? It seems several groups introduced independently this technique}. The basic idea is to allow the program to take on ``symbolic'' -- instead of concrete -- input values. Variables and control flow paths are associated with expressions and constraints in terms of those symbols during a symbolic execution of the program, and constraints are eventually solved via SMT (satisfiability modulo theories) solvers.

In this article we survey the main aspects of symbolic execution and discuss its extensive usage in computer security applications\mynote{IF: want focus on security?}, where software vulnerabilities can be found by symbolically executing programs at the level of either source or binary code.
We start with a simple example that will introduce many of the fundamental issues discussed in remainder of the paper.

\subsection{Warm-up example}
\label{symbolic-execution-example}

Consider the simple C function shown in Figure~\ref{fig:example-1}, whose most critical operation is the division at line 10. Each of the input values {\tt a}, {\tt b}, and {\tt c} can be assigned with $2^{32}$ distinct integer values, though only values leading to $x + y + z - 3 = 0$ make the code crash. While techniques such as random testing could generate bottomless input tests for this function, it is unlikely that exactly the crash-inducing inputs would be randomly picked up\mynote{Fuzzing?}. 
Symbolic execution overcomes these limitations by evaluating a piece of code using {\em symbols}, instead of concrete values, for its inputs. This makes it possible to reason on {\em classes of input values}, instead of single input instances. 

\begin{figure}[t]
\begin{lstlisting}[basicstyle=\ttfamily\small]
              1.  int foobar(int a, int b, int c) {
              2.    int x = 0, y = 0, z = 0;
              3.    if (a != 0)
              4.      x = -2;
              5.    if (b < 5) {
              6.      z = 2;
              7.      if (a == 0 && c != 0)
              8.        y = 1;
              9.    }
             10.    return a / (x + y + z - 3);
             11.  }
\end{lstlisting}
\caption{Simple C function used in the warm-up example.}
\label{fig:example-1}
\end{figure}

In more details, every global or local variable is associated with a symbol $\alpha_i$.  At any point of the execution, the symbolic engine maintains:

\begin{itemize}
  \item  an execution state $(stmt,~pc)$ where:
\begin{itemize}
  \item $stmt$ is the statement to evaluate. For the time being we assume that $stmt$ can be an assignment, a conditional branch or a jump (a discussion of more complex constructs for iteration and function calls will be provided in Section~\ref{example-discussion});
  \item $pc$ is a collection of path\mynote{IF: my feeling is that pc does not denote a set. Better notation? Isn't pc a logical formula?} constraints, i.e., a set of assumptions made over the symbols $\alpha_i$ in order to reach $stmt$. Initially $pc$ is empty, and is thus trivially satisfied.
\end{itemize}
\item a {\em memory mapping} $M$ for storing constraints over different symbols. This is required since in general every memory location can be associated with a symbol\mynote{Rephrase}.
\end{itemize}

\noindent Depending on $stmt$, the symbolic engine modifies the state as follows:
\begin{itemize}
  \item $stmt$ is a constant assignment $\alpha_i = c$: when a constant value $c$ is assigned to a variable associated to the symbol $\alpha_i$, $pc$ is extended by adding a constraint on $\alpha_i$:\mynote{Redundant? IF: I would remove the constant assignment}
    \[ pc \gets pc \wedge \alpha_i = c\]

  \item $stmt$ is an assignment $\alpha_i = e$: when an expression $e$ is assigned to a symbol $\alpha_i$, $pc$ is extended by adding a constraint on $\alpha_i$:
    \[ pc \gets pc \wedge \alpha_i = e\]
  where $e$ can be any expression, involving unary or binary operators, over symbols and constants.

  \item $stmt $ is a conditional branch ${\tt if}~e~{\tt then}~s_{true}~{\tt else}~s_{false}$: $pc$ is evaluated. Two scenarios are possible:
    \begin{itemize}
      \item (non-forking) $e$ is evaluated as always true or false under the assumptions in $pc$: the proper branch is taken, and symbolic execution advances to $s_{true}$ or $s_{false}$ accordingly;
      \item (forking) $e$ cannot be evaluated without instantiating values for one or more symbols in it: the symbolic execution process is forked, creating two execution states:
        \[ (s_{true}, pc_{true}) \text{ where } pc_{true} = pc \wedge e \]
        \[ (s_{false}, pc_{false}) \text{ where } pc_{false} = pc \wedge \neg e \]
    \end{itemize}
    Symbolic execution proceeds on both states in parallel.

  \item $stmt $ is a jump {\tt goto} $s$: execution state is updated to advance symbolic execution to $s$. 
\end{itemize}

%\subsection{Example}
%\label{symbolic-execution-example}

\begin{figure}[t]
  \centering
  \includegraphics[width=1.0\columnwidth]{images/example} 
  \caption{Symbolic execution tree of the function {\tt foobar}. Each execution state is labeled with an alphabet letter. Side effects on execution states are highlighted in gray. Leaves are evaluated against division by zero error. For the sake of presentation the conjunction of constraints is shown as a list of constraints. }
  \label{fig:example-symbolic-execution}
\end{figure}

A symbolic execution of the function {\tt foobar} is shown in Figure~\ref{fig:example-symbolic-execution}. Initially, a new symbol is introduced for each input argument and for each local variable. For the sake of the presentation, we associate a symbol $\alpha_{var}$ with each variable $var$. 
Moreover, we assume that the set of local\mynote{locals only right? I added "at each point"} variables in use at each point by the function is known. In general, obtaining this information may be non trivial, and symbols are typically introduced when statements defining the variables are reached.
%Moreover, we have assumed to know the set of local variables used by the function. In general, obtaining this information may be non trivial and thus it it common to introduce symbols related to local variables only when the statements defining the variables are evaluated.
We also maintain a mapping table to track the mapping between variables and symbols. 

The first statement of the function is located at line 2. For this reason, the initial execution state $A$ is given by $(2, \{\})$ where the path constraint set is empty as no assumption is made on any symbol. After executing line 2,  $pc$ is updated by adding constraints on the value of $\alpha_x$, $\alpha_y$, and $\alpha_z$ (execution state $B$). Line 3 contains a conditional branch: since the condition cannot be uniquely determined based on the current set of assumptions, the execution is forked. Depending on the branch taken, a different statement is evaluated next and different assumptions are made on the symbol (execution states $C$ and $D$). On the other hand, when a condition can be uniquely determined (e.g., as in execution state $H$), it is sufficient to follow only the relevant branch (e.g., going to state $L$ from $H$), thus pruning unrealistic execution states. 

After expanding every execution state until the statement on line 10 is reached, we can check which input values for parameters {\tt a}, {\tt b}, and {\tt c} can make the function {\tt foobar} crash as a consequence of a division-by-zero operation. Analyzing execution states $\{L, M, N, O, P\}$, we can conclude that only $N$ can lead to an unsafe operation. The path constraint set for $N$ thus defines\mynote{implicitly defines?} the set of inputs that are unsafe for {\tt foobar}. Indeed, any input values for $\alpha_a$, $\alpha_b$, and $\alpha_c$ such that:
 \[ \alpha_a = 0 \wedge \alpha_b < 5 \wedge \alpha_c \neq 0 \]
will make the function crash. An instance of unsafe input parameters for {\tt foobar} can determined exploiting a constraint solver (i.e., an oracle able to resolve constraints). For instance, given the execution state $N$, a solver may come up with the values $a = 0$, $b = 1$, and $c = 0$. Notice\mynote{Say earlier?} that a constraint solver is also needed when evaluating the satisfiability of branch conditions.

\subsection{Discussion}
\label{example-discussion}

The example described in Section~\ref{symbolic-execution-example} shows the effectiveness of symbolic execution in identifying {\em all} the possible unsafe input values that can trigger a crash due to an unsafe division performed at line 10. This is achieved through an exhaustive exploration of all the possible execution states. For this reason, symbolic execution is a sound and complete methodology from a theoretical perspective. Soundness guarantees that input values deemed as unsafe are actually unsafe, and completeness implies that all possible unsafe inputs will be found. However, challenges that symbolic execution has to face when processing real-world code can be significantly more complex compared to those from our toy example of Figure~\ref{fig:example-1}. Several observations and questions naturally arise:

\begin{enumerate}

  \item (objects) {\em How does symbolic execution handle arrays or other more complex objects?} \\
  In general, any arbitrary complex object can be seen as an array of bytes, where each byte is associated with a distinct symbol. In principle, even a C {\tt int} variable can be seen as an array of four bytes.\mynote{Cosa intendiamo qui?} However, it is convenient to exploit structural properties of the data when possible (e.g., by statically analyzing the source code). For instance, for object-oriented languages the search performed by symbolic execution can be refined taking advantage of relational bounds on class fields.

  \item (loops) {\em How does symbolic execution handle loops?} \\
  In the execution model presented in Section~\ref{simple-execution-model} a loop can be encoded as a combination of conditional branches and $goto$ statements. This transformation is frequent when lowering a high-level language like C to an intermediate representation or native code. When the number of loop iterations cannot be determined in advance (e.g., it depends on an input parameter), for a symbolic execution engine choosing how many iterations should be analyzed becomes critical. The naive approach of unrolling iterations for every valid index bound leads to a very large number of states. It is possible to limit the number of iterations to $k$, thus trading speed for completeness, or when loop invariants can be inferred through static analysis they can be used to merge equivalent states (\mynote{Recuperare citazione}e.g., when differences are not observable outside the loop body).

  \item (subroutines) {\em How does symbolic execution handle subroutines?} \\
  Our execution model does not handle invocation of subroutines (i.e., a $call$ statement). \mynote{Extend this paragraph} A way to extend it to support subroutines is to provide the execution state with a simple execution stack.

  \item (recursion) {\em How does symbolic execution handle recursion?} \\
  Consider, as an example, the code:
    \begin{lstlisting}[basicstyle=\ttfamily\small]
    1.  int bar(int n) {
    2.    if (n >= 0) 
    3.      return 0;
    4.    return 1 + sum(n - 1);
    5.  }
    \end{lstlisting}
  Assuming that an execution stack has been added to the state, we observe that this code can easily lead to a very large number of execution states (i.e., a new state is subsequently created every time the branch on line 2 is not taken). As an {\tt int} variable can have up to  $2^{31} - 1$ positive values, symbolic execution has to create as many execution states to cover all the possible execution paths.
 %Indeed, the number of executions states is related to the number of times that the conditional branch on line 2 is not taken. 
 
  \item (environment) {\em How does symbolic execution handle interaction with the environment}? \\
  Real-world applications interacts constantly with the environment (e.g., filesystem, network) through libraries or system calls. A crucial aspect of these interactions is that they may cause side-effects
% on the environments 
(e.g., creation of a file)
%or initialization of a memory area
that must be taken into account, as they may later affect the 
%actual
execution of the code. Evaluating any possible outcome of an interaction is typically not feasible due to the large number of possibly generated execution states, only a small number of which can actually happen in a non-symbolic scenario. Hence, it is common to create models for popular library and system routines that help the symbolic execution engine to consider only significant outcomes.

  \item (state space explosion, path selection) {\em How does symbolic execution deal with path explosion}? \\
  A relatively simple code such as function {\tt foobar}, which is composed by less 12 lines of code, has generated 16 execution states, where $5$ out of $16$ are independent\mynote{Independent?} and must be checked to determine possible unsafe input values. Although this could seem a reasonable number of states, language constructs such as loops may contribute to increase the number of states exponentially. For this reason, it is unlikely that a symbolic execution engine is able to exhaustively explore all the possible execution states within a reasonable amount of time. In practice, heuristics are used to guide exploration and prioritize certain states first (e.g., to maximize code coverage), hoping this would lead to interesting discoveries. Also, a symbolic execution engine should implement efficient mechanism for evaluating multiple execution states in parallel without running out of resources.
  %In practice, several heuristics must be exploited to prioritize evaluation of some states, hoping to still be able to spot interesting things. Moreover, the symbolic execution engine should include efficient mechanism for efficiently evaluating in parallel different execution states without running out of computational resources.

  \item (constraint solver) {\em What can a constraint solver do in practice?}
  %{\em What is a constraint solver in practice}? \\
 Constraint solvers suffer from a number of limitations. Typically, they can handle complex constraints in a reasonable amount of time only if they are made of linear expressions over their constituents. Symbolic execution engines typically implement a number of optimizations to make queries as much {\em solver-friendly} as possible, for instance by splitting queries in independent components to process separately or by performing algebraic simplifications.

  \item (binary code) {\em What are the disadvantages of symbolically executing binary code}? \\
  The example presented in Section~\ref{symbolic-execution-example} is written in C. This does not imply that symbolic execution cannot be performed directly on binary code, which in several scenarios is the only available representation of a program. However, having the source code of an application makes symbolic execution significantly easier, as it can exploit high-level properties (e.g., object shapes) that can be inferred by statically analyzing the source code.
  %(e.g., the maximum size of a buffer or the number of iterations for a loop).
   
\end{enumerate}
%Depending on the specific application context of symbolic execution
Depending on the specific context in which symbolic execution is used, different choices and assumptions are made to address the questions highlighted above. Although they typically affect soundness or completeness, in several scenarios a partial exploration of the space of possible execution states is typically sufficient to reach the goal\mynote{Better example?} (e.g., identify a crashing input for an application) with a limited time budget.

%different choices and assumptions are made to address the above questions. Although soundness and completeness of symbolic execution may be negatively affected by these choices, there are several application scenarios where a partial exploration of the possible execution states is sufficient for reaching the ultimate goal (e.g., identify a single input that crashes an application).

\subsection{Paper organization}

%\vspace{2cm}
%\subsection{Removed stuff}
%
%\paragraph{Black-box approach versus white-box approach}
%
%Discussion\mynote{IF: do we really need this?} of black-box approach and white-box approach. Symbolic execution is a white-box technique. Black-box approaches can be very fast but not always effective. White-box approaches can be very effective but are typically slower than black-box techniques. An in-depth discussion of this aspect will be done when we will discuss~\cite{DRILLER-NDSS16}.
%
%\begin{figure}[H]
%  \vspace{-3mm}
%  \centering
%  \begin{subfigure}{.5\textwidth}
%    \centering
%    \includegraphics[width=0.9\linewidth]{images/blackbox} 
%    \caption{Black-box approach}
%    %\label{fig:sub1}
%  \end{subfigure}%
%  \begin{subfigure}{.5\textwidth}
%    \centering
%    \includegraphics[width=0.9\linewidth]{images/whitebox} 
%    \caption{White-box approach}
%    %\label{fig:sub2}
%  \end{subfigure}
%  %\label{fig:example-symbolic-execution}
%  \vspace{-3mm}
%\end{figure}
%
%\paragraph{Taken from old Overview}
%
%Symbolic execution has been originally introduced in~\cite{K-CACM76} and~\cite{H-TSE77}. A good introduction to symbolic execution is presented in~\cite{KLEE-OSDI08}.\mynote{Extend this paragraph}
%%(while~\cite{EXE-CCS06} is a previous effort of the same authors).
%\cite{SAGE-NDSS08} is one successful story of symbolic execution. \cite{SAB-SP10} presents a neat formalization of symbolic execution and of taint analysis as well.
%

% \myinput{executors}
% \myinput{memory}
% \myinput{environment}
% \myinput{loops}
% \myinput{explosion}
% \myinput{constraints}
% \myinput{binary}
% % !TEX root = appendix.tex

\section{Sample Applications}
\label{se:applications}

%\revedit{
The last decade has witnessed an increasing adoption of symbolic execution techniques not only in the software testing domain, but also to address other compelling engineering problems such as automatic generation of exploits or authentication bypass. We now discuss \iffullver{three prominent}{prominent} applications of symbolic execution techniques to these domains. Examples of extensions to other areas can be found, e.g., in~\cite{CGK-ICSE11}.
%}

%The last decade has witnessed an increasing adoption of symbolic execution techniques not only in the software testing domain, but also to address other compelling engineering problems such as automatic generation of exploits or authentication bypass. We now discuss \iffullver{three prominent}{prominent} applications of symbolic execution techniques to these domains. Examples of extensions to other areas can be found, e.g., in~\cite{CGK-ICSE11}.

\subsection{Bug Detection}\mynote{Rendere piu' di ampio respiro il titolo di questa sezione? Keyword: software testing, program understanding}
\label{ss:bug-detection}

Software testing strategies typically attempt to execute a program with the intent of finding bugs. As manual test input generation is an error-prone and usually non-exhaustive process, automated testing technique have drawn a lot of attention over the years. Random testing techniques such as fuzzing are cheap in terms of run-time overhead, but fail to obtain a wide exploration of a program state space. Symbolic and concolic execution techniques on the other hand achieve a more exhaustive exploration, but they become expensive as the length of the execution grows: for this reason, they usually reveal shallow bugs only.

\cite{RK-ICSE07} proposes {\em hybrid concolic testing} for test input generation, which combines random search and concolic execution to achieve both deep program states and wide exploration. The two techniques are interleaved: in particular, when random testing saturates (i.e., it is unable to hit new code coverage points after a number of steps), concolic execution is used to mutate the current program state by performing a bounded depth-first search for an uncovered coverage point. For a fixed time budget, the technique outperforms both random and concolic testing in terms of branch coverage. The intuition behind this approach is that many programs show behaviors where a state can be easily reached through random testing, but then a precise sequence of events -- identifiable by a symbolic engine -- is required to hit a specific coverage point.

% which uses preconstraining on the program states to ensure consistency
\cite{DRILLER-NDSS16} refines this idea and devises a vulnerability excavation tool based on {\sc Angr}~\cite{ANGR-SSP16}, called Driller, that interleaves fuzzing and concolic execution to discover memory corruption vulnerabilities. The authors remark that user inputs can be categorized as {\em general} input, which has a wide range of valid values, and {\em specific} input: a check for particular values of a specific input then splits an application into {\em compartments}. Driller offloads the majority of unique path discovery to a fuzzy engine, and relies on concolic execution to move across compartments. During the fuzzy phase, Driller marks a number of inputs as interesting (for instance, when an input was the first to trigger some state transition) and once it gets stuck in the exploration, it passes the set of such paths to a concolic engine, which preconstraints the program states to ensure consistency with the results of the native execution. On the dataset used for the DARPA Cyber Grand Challenge qualifying event, Driller could identify crashing inputs in 77 applications, including both the 68 and 16 applications for which fuzzing and symbolic execution alone succeeded, respectively. For 6 applications, Driller was the only one to detect a vulnerability.

% temporaneamente messo qui
\mytempedit{
	Maintenance of large and complex applications is a very hard task. Fixing bugs can sometimes even introduce new and unexpected issues in the software, which in turn may require several hours or even weeks to be detected and properly addressed by the developers. \cite{QRL-TOSEM12} tackles the problem of identifying the root cause of failures during regression testing. Given a program $P$ and a newer revision of the program $P'$, if a testing input $t$ generates a failure in $P'$ but not in  $P$, then symbolic execution is used to track the path constraints $\pi$ and $\pi'$ when executing $P$ and $P'$ on the failing input $t$, respectively. Using a SMT solver, a new input $t'$ is generated by solving the constraint $\pi ~\wedge \neg\pi'$. If $t'$ exists (i.e., the constraint is satisfiable), then $P'$ has one or more {\em deviations} in the control-flow graph with respect to $P$ that can be the root cause of the failure. By carefully tracking branch conditions during symbolic execution, \cite{QRL-TOSEM12} are even able to pinpoint which branches are responsible for these deviations. If $\pi \wedge \neg\pi'$ is not satisfiable, then the symmetric constraint query $\neg\pi \wedge \pi'$ is tested and a similar reasoning is performed to detect the possible branch conditions that may have lad to the failure. If $\neg\pi \wedge \pi'$ is also unsatisfiable, then \cite{QRL-TOSEM12} cannot determine the root cause of the problem.

	Another interesting work that targets the problem of software regressions through the use of symbolic execution is~\cite{BOR-ICSE13}. This work introduce an approach called {\em partition-based regression verification} that combines the advantages of both regression verification (RV) and regression testing (RT). Indeed, RV is a very powerful technique for identifying regressions but hardly scales over large programs due to the difficulty in proving behavioral equivalence between the original and the modified program. On the other hand, RT allows to check a modified program for regressions by testing selected concrete sample inputs, making it more scalable but providing limited verification guarantees. The main intuition behind partition-based regression verification is to identify {\em differential partitions}. Each differential partition can be seen as a subset of the input space for which the two program versions -- given the same path constraints -- either expose the same output ({\em equivalence-revealing partition}) or produce different outputs ({\em difference-revealing partition}). For each partition, a test case is generated and added to the regression test suite, which can later be used by a developer for classical RT. Since differential partitions are derived exploiting symbolic execution, this approach suffers from the common limitations that come with this technique. However, if the exploration is interrupted (e.g., due to excessive time or memory usage), partition-based regression verification can still provide guarantees over the subset of input space that has been covered by the detected partitions.

	Static data flow analysis tools can significantly help developers tracks malicious data leaks in software applications. Unfortunately, they often report several allegedly bugs that only after manual inspection can be regarded as false positives. To mitigate this issue,~\cite{ARH-SOAP15} has proposed TASMAN, a system that, after performing data flow analysis to track information leaks, uses symbolic backward execution (SBE) to test each reported bug. Starting from a leaking statement, TASMAN backwards into the code, pruning any path that can be proved to be unfeasible. If all the paths starting from the leaking statement are discarded by TASMAN, then the reported bug can be marked as a false positive.

	Although symbolic execution has been extensively used for bug detection, during the last decades several works~\cite{GDV-ISSTA12,FPV-ICSE13,CLL-ICSE16} have shown how it can be also used for other program understanding activities. For instance,~\cite{GDV-ISSTA12} has introduced {\em probabilistic symbolic execution}, an approach that makes it possible to compute the probability of executing different code portions of a program. This is achieved by exploiting model counting techniques, such as the {\tt LattE}~\cite{LHT-JSC04} toolset, that allows~\cite{GDV-ISSTA12}  to determine the number of solutions for the different path constraints given by the alternative execution paths of a program. The paper by~\cite{FPV-ICSE13} makes a step further and uses probabilistic symbolic execution for performing software reliability analysis. This is computed as the probability of executing any path that has been labeled as successful given a usage profile. Intuitively, a usage profile can be seen as the distribution over the input space. Since in general the termination of symbolic execution cannot be guaranteed in presence of loops, then~\cite{FPV-ICSE13} resorts to bounded exploration. Nonetheless, they define a metric for evaluating the confidence in their reliability estimation, allowing a developer to increase the bounds in order to improve the confidence value. Of a different flavor is the work by~\cite{CLL-ICSE16} that exploits probabilistic symbolic execution to conduct performance analysis. Based on usage profiles and on path execution probabilities, paths are classified into two types: {\em low probability} and {\em high probability}. In a first phase, high-probability paths are explored in a way that maximizes path diversity, generating a first set of test inputs. In the second phase, low-probability paths are analyzed using symbolic execution, generating a second set of test inputs that should expose executions characterized by best-execution times and by worst-execution times. Finally, the program is executed using the test inputs generated during the two phases and running times are measured to generate performance distributions. 

	Another interesting application of symbolic execution is presented by~\cite{PPM-CSF18}. Their technique exploits model counting and symbolic execution for computing quantitative bounds on the amount of information that can be leaked by a program through side-channel attacks. 
	
}
%As it is based on {\sc Angr}, Driller adopts an index-based memory model as in Section~\ref{ss:index-based-memory} where reads can be symbolic and writes are always concretized. % read/write addresses

\subsection{Bug Exploitation}
\label{ss:bug-exploitation}
Bugs are a consequence of the nature of human factors in software development and are everywhere. Those that can be exploited by an attacker should normally be fixed first: systems for automatically and effectively identifying them are thus very valuable.

{\sc AEG}~\cite{AEG-NDSS11} employs preconditioned symbolic execution to analyze a potentially buggy program in source form and look for bugs amenable to stack smashing or return-into-libc exploits~\cite{PB-SSP04}, which are popular control hijack attack techniques. The tool augments path constraints with exploitability constraints and queries a constraint solver, generating a concrete exploit when the constraints are satisfiable. The authors devise the {\em buggy-path-first} and {\em loop-exhaustion} strategies (Table~\ref{tab:heuristics}) to prioritize paths in the search. On a suite of 14 Linux applications, {\sc AEG} discovered 16 vulnerabilities, 2 of which were previously unknown, and constructed control hijack exploits for them.

{\sc Mayhem}~\cite{MAYHEM-SP12} takes another step forward by presenting the first system for binary programs that is able identify end-to-end exploitable bugs. It adopts a hybrid execution model based on checkpoints and two components: a concrete executor that injects taint-analysis instrumentation in the code and a symbolic executor that takes over when a tainted branch or jump instruction is met. Exploitability constraints for symbolic instruction pointers and format strings are generated, targeting a wide range of exploits, e.g., SEH-based and jump-to-register ones. Three path selection heuristics help prioritizing paths that are most likely to contain vulnerabilities (e.g., those containing symbolic memory accesses or instruction pointers). A virtualization layer intercepts and emulates all the system calls to the host OS, while preconditioned symbolic execution can be used to reduce the size of the search space. Also, restricting symbolic execution to tainted basic blocks only gives very good speedups in this setting, as in the reported experiments more than $95\%$ of the processed instructions were not tainted. {\sc Mayhem} was able to find exploitable vulnerabilities in the 29 Linux and Windows applications considered in the evaluation, 2 of which were previously undocumented. Although the goal in {\sc Mayhem} is to reveal exploitable bugs, the generated simple exploits can be likely transformed in an automated fashion to work in the presence of classical OS defenses such as data execution prevention and address space layout randomization~\cite{Q-SEC11}. 

\vspace{-1mm} % TODO
\subsection{Authentication Bypass}
\label{ss:auth-bypass}
Software backdoors are a method of bypassing authentication in an algorithm, a software product, or even in a full computer system. Although sometimes these software flaws are injected by external attackers using subtle tricks such as compiler tampering~\cite{KRS-TR74}, there are reported cases of backdoors that have been surreptitiously installed by the hardware and/or software manufacturers~\cite{CZF-USEC14}, or even by governments~\cite{NSA-BACKDOOR}. 

Different works~\cite{DMR-USEC13,ZBF-NDSS14,FIRMALICE-NDSS15} have exploited symbolic execution for analyzing the behavior of binary firmwares. Indeed, an advantage of this technique is that it can be used even in environments, such as embedded systems, where the documentation and the source code that are publicly released by the manufacturer are typically very limited or none at all. For instance,~\cite{FIRMALICE-NDSS15} proposes Firmalice, a binary analysis framework based on {\sc Angr}~\cite{ANGR-SSP16} that can be effectively used for identifying authentication bypass flaws inside firmwares running on devices such as routers and printers. Given a user-provided description of a privileged operation in the device, Firmalice identifies a set of program points that, if executed, forces the privileged operation to be performed. The program slice that involves the privileged program points is then symbolically analyzed using {\sc Angr}. If any such point can be reached by the engine, a set of concrete inputs is generated using an SMT solver. These values can be then used to effectively bypass authentication inside the device. On three commercially available devices, Firmalice could detect vulnerabilities in two of them, and determine that a backdoor in the third firmware is not remotely exploitable.
% % !TEX root = paper.tex
\section{Conclusions}
\label{se:conclusions}

Techniques for symbolic execution have evolved significantly in the last decade, leading to major practical breakthroughs. In 2016, the DARPA Cyber Grand Challenge hosted systems that can detect and fix vulnerabilities in unknown software with no human intervention, such as {\sc Angr}~\cite{ANGR-SSP16} and {\sc Mayhem}~\cite{MAYHEM-SP12}, which won the \$2M first prize. {\sc Mayhem} was also the first autonomous software to play the Capture-The-Flag contest at the DEF CON 24 hacker convention. The event demonstrated that tools for automatic exploit detection based on symbolic execution can be competitive with human experts, paving the road to unprecedented applications and the rise of start-ups that have the potential to shape software security and reliability in the next decades. 

For details on the techniques behind symbolic executors, we would like to suggest the interested reader to have a look at a survey we have recently completed. The survey, available at \url{https://arxiv.org/abs/1610.00502}, has been written with the goal of providing an overview of the main ideas, challenges, and solutions developed in the area, distilling them for a broad audience.

\iffalse

Techniques for symbolic execution have evolved significantly in the last decade, leading to major practical breakthroughs. In 2016, the DARPA Cyber Grand Challenge hosted systems that can detect and fix vulnerabilities in unknown software with no human intervention, such as {\sc Angr}~\cite{ANGR-SSP16} and {\sc Mayhem}~\cite{MAYHEM-SP12}, which won the \$2M first prize. {\sc Mayhem} was also the first autonomous software to play the Capture-The-Flag contest at the DEF CON 24 hacker convention\footnote{\url{https://www.defcon.org/html/defcon-24/dc-24-ctf.html}.}. The event demonstrated that tools for automatic exploit detection based on symbolic execution can be competitive with human experts, paving the road to unprecedented applications %and the rise of start-ups 
that have the potential to shape software %security and 
reliability in the next decades. 

This survey has discussed some of the key aspects and challenges of symbolic execution, presenting them for a broad audience. To explain the basic design principles of symbolic executors and the main optimization techniques, we have focused on single-threaded applications with integer arithmetic. Symbolic execution of multi-threaded programs is treated, e.g., \iffullver{in~\cite{KPV-TACAS03,SA-HVC06,CLOUD9-EUROSYS11,FHR-ESEC13,BGC-OOPSLA14,GKW-ESEC15}}{in~\cite{FHR-ESEC13,BGC-OOPSLA14,GKW-ESEC15}}, while techniques for programs that manipulate floating point data are addressed \iffullver{in, e.g., \cite{M-STVR01,BGM-STVR06,LTH-ICTSS10,CCK-EUROSYS11,BVL-POPL13,CCK-TSE14,RPW-SIGSOFT15}}{in, e.g., \cite{BVL-POPL13,CCK-TSE14,RPW-SIGSOFT15}}.

We hope that this survey will help non-experts grasp the key inventions in the exciting line of research of symbolic execution, inspiring further work and new ideas.

\ifdefined\arxivver

\myparagraph{Acknowledgements}
This work is supported in part by a grant of the Italian Presidency of the Council of Ministers and by the CINI (Consorzio Interuniversitario Nazionale Informatica) National Laboratory of Cyber Security.

\myparagraph{Live Version of this Article}
We complement the traditional scholarly publication model by maintaining a live version of this article at {\href{https://github.com/season-lab/survey-symbolic-execution}{https://github.com/season-lab/survey-symbolic-execution/}}. The live version incorporates continuous feedback by the community, providing post-publication fixes, improvements, and extensions.
\fi

\fi

% % !TEX root = main.tex

\section{Glossary}
\label{se:glossary}

\noindent {\bf Complete analysis.} Analysis that guarantees no false positives, i.e., all reported property violations are true.

\smallskip\noindent {\bf Concrete execution.} An execution of a program using concrete inputs in a real-world environment.

\smallskip\noindent {\bf Concolic execution.} \ldots

\smallskip\noindent {\bf Control flow graph (CFG).} Representation of a program that uses nodes to model instructions and edges to model the control flow between them.

\smallskip\noindent {\bf Control flow path.} Path in the control flow graph of a program. Represents the sequence of instructions executed by the program for a given concrete input.

\smallskip\noindent {\bf Decidable analysis} \ldots

\smallskip\noindent {\bf Model checker.} Given a model of a system, a model checker exhaustively and automatically checks whether the model meets a given specification.

\smallskip\noindent {\bf Symbolic execution.} \ldots

\smallskip\noindent {\bf Symbolic store.} \ldots

\smallskip\noindent {\bf SMT solver.} A Satisfiability Modulo Theories (SMT) instance is a formula in first-order logic, where some function and predicate symbols have additional interpretations, and SMT is the problem of determining whether such a formula is satisfiable. A SMT solver is a tool able to reason over SMT formulas.

\smallskip\noindent {\bf Sound analysis.} Analysis that guarantees no false negatives, i.e., if there is a property violation, then it is reported.

\smallskip\noindent {\bf Path constraints.} \ldots

% !TEX root = CSUR/main.tex

\section{Introduction}

Symbolic execution is a powerful program analysis technique introduced in the mid 70's in the context of software testing (see, e.g.,~{\cite{K-CACM76} and~\cite{H-TSE77}})\mynote{IF: are references appropriate? Should we give more credits? It seems several groups introduced independently this technique}. The basic idea is to allow the program to take on ``symbolic'' -- instead of concrete -- input values. Variables and control flow paths are associated with expressions and constraints in terms of those symbols during a symbolic execution of the program, and constraints are eventually solved via SMT (satisfiability modulo theories) solvers.

In this article we survey the main aspects of symbolic execution and discuss its extensive usage in computer security applications\mynote{IF: want focus on security?}, where software vulnerabilities can be found by symbolically executing programs at the level of either source or binary code.
We start with a simple example that will introduce many of the fundamental issues discussed in remainder of the paper.

\subsection{Warm-up example}
\label{symbolic-execution-example}

Consider the simple C function shown in Figure~\ref{fig:example-1}, whose most critical operation is the division at line 10. Each of the input values {\tt a}, {\tt b}, and {\tt c} can be assigned with $2^{32}$ distinct integer values, though only values leading to $x + y + z - 3 = 0$ make the code crash. While techniques such as random testing could generate bottomless input tests for this function, it is unlikely that exactly the crash-inducing inputs would be randomly picked up\mynote{Fuzzing?}. 
Symbolic execution overcomes these limitations by evaluating a piece of code using {\em symbols}, instead of concrete values, for its inputs. This makes it possible to reason on {\em classes of input values}, instead of single input instances. 

\begin{figure}[t]
\begin{lstlisting}[basicstyle=\ttfamily\small]
              1.  int foobar(int a, int b, int c) {
              2.    int x = 0, y = 0, z = 0;
              3.    if (a != 0)
              4.      x = -2;
              5.    if (b < 5) {
              6.      z = 2;
              7.      if (a == 0 && c != 0)
              8.        y = 1;
              9.    }
             10.    return a / (x + y + z - 3);
             11.  }
\end{lstlisting}
\caption{Simple C function used in the warm-up example.}
\label{fig:example-1}
\end{figure}

In more details, every global or local variable is associated with a symbol $\alpha_i$.  At any point of the execution, the symbolic engine maintains:

\begin{itemize}
  \item  an execution state $(stmt,~pc)$ where:
\begin{itemize}
  \item $stmt$ is the statement to evaluate. For the time being we assume that $stmt$ can be an assignment, a conditional branch or a jump (a discussion of more complex constructs for iteration and function calls will be provided in Section~\ref{example-discussion});
  \item $pc$ is a collection of path\mynote{IF: my feeling is that pc does not denote a set. Better notation? Isn't pc a logical formula?} constraints, i.e., a set of assumptions made over the symbols $\alpha_i$ in order to reach $stmt$. Initially $pc$ is empty, and is thus trivially satisfied.
\end{itemize}
\item a {\em memory mapping} $M$ for storing constraints over different symbols. This is required since in general every memory location can be associated with a symbol\mynote{Rephrase}.
\end{itemize}

\noindent Depending on $stmt$, the symbolic engine modifies the state as follows:
\begin{itemize}
  \item $stmt$ is a constant assignment $\alpha_i = c$: when a constant value $c$ is assigned to a variable associated to the symbol $\alpha_i$, $pc$ is extended by adding a constraint on $\alpha_i$:\mynote{Redundant? IF: I would remove the constant assignment}
    \[ pc \gets pc \wedge \alpha_i = c\]

  \item $stmt$ is an assignment $\alpha_i = e$: when an expression $e$ is assigned to a symbol $\alpha_i$, $pc$ is extended by adding a constraint on $\alpha_i$:
    \[ pc \gets pc \wedge \alpha_i = e\]
  where $e$ can be any expression, involving unary or binary operators, over symbols and constants.

  \item $stmt $ is a conditional branch ${\tt if}~e~{\tt then}~s_{true}~{\tt else}~s_{false}$: $pc$ is evaluated. Two scenarios are possible:
    \begin{itemize}
      \item (non-forking) $e$ is evaluated as always true or false under the assumptions in $pc$: the proper branch is taken, and symbolic execution advances to $s_{true}$ or $s_{false}$ accordingly;
      \item (forking) $e$ cannot be evaluated without instantiating values for one or more symbols in it: the symbolic execution process is forked, creating two execution states:
        \[ (s_{true}, pc_{true}) \text{ where } pc_{true} = pc \wedge e \]
        \[ (s_{false}, pc_{false}) \text{ where } pc_{false} = pc \wedge \neg e \]
    \end{itemize}
    Symbolic execution proceeds on both states in parallel.

  \item $stmt $ is a jump {\tt goto} $s$: execution state is updated to advance symbolic execution to $s$. 
\end{itemize}

%\subsection{Example}
%\label{symbolic-execution-example}

\begin{figure}[t]
  \centering
  \includegraphics[width=1.0\columnwidth]{images/example} 
  \caption{Symbolic execution tree of the function {\tt foobar}. Each execution state is labeled with an alphabet letter. Side effects on execution states are highlighted in gray. Leaves are evaluated against division by zero error. For the sake of presentation the conjunction of constraints is shown as a list of constraints. }
  \label{fig:example-symbolic-execution}
\end{figure}

A symbolic execution of the function {\tt foobar} is shown in Figure~\ref{fig:example-symbolic-execution}. Initially, a new symbol is introduced for each input argument and for each local variable. For the sake of the presentation, we associate a symbol $\alpha_{var}$ with each variable $var$. 
Moreover, we assume that the set of local\mynote{locals only right? I added "at each point"} variables in use at each point by the function is known. In general, obtaining this information may be non trivial, and symbols are typically introduced when statements defining the variables are reached.
%Moreover, we have assumed to know the set of local variables used by the function. In general, obtaining this information may be non trivial and thus it it common to introduce symbols related to local variables only when the statements defining the variables are evaluated.
We also maintain a mapping table to track the mapping between variables and symbols. 

The first statement of the function is located at line 2. For this reason, the initial execution state $A$ is given by $(2, \{\})$ where the path constraint set is empty as no assumption is made on any symbol. After executing line 2,  $pc$ is updated by adding constraints on the value of $\alpha_x$, $\alpha_y$, and $\alpha_z$ (execution state $B$). Line 3 contains a conditional branch: since the condition cannot be uniquely determined based on the current set of assumptions, the execution is forked. Depending on the branch taken, a different statement is evaluated next and different assumptions are made on the symbol (execution states $C$ and $D$). On the other hand, when a condition can be uniquely determined (e.g., as in execution state $H$), it is sufficient to follow only the relevant branch (e.g., going to state $L$ from $H$), thus pruning unrealistic execution states. 

After expanding every execution state until the statement on line 10 is reached, we can check which input values for parameters {\tt a}, {\tt b}, and {\tt c} can make the function {\tt foobar} crash as a consequence of a division-by-zero operation. Analyzing execution states $\{L, M, N, O, P\}$, we can conclude that only $N$ can lead to an unsafe operation. The path constraint set for $N$ thus defines\mynote{implicitly defines?} the set of inputs that are unsafe for {\tt foobar}. Indeed, any input values for $\alpha_a$, $\alpha_b$, and $\alpha_c$ such that:
 \[ \alpha_a = 0 \wedge \alpha_b < 5 \wedge \alpha_c \neq 0 \]
will make the function crash. An instance of unsafe input parameters for {\tt foobar} can determined exploiting a constraint solver (i.e., an oracle able to resolve constraints). For instance, given the execution state $N$, a solver may come up with the values $a = 0$, $b = 1$, and $c = 0$. Notice\mynote{Say earlier?} that a constraint solver is also needed when evaluating the satisfiability of branch conditions.

\subsection{Discussion}
\label{example-discussion}

The example described in Section~\ref{symbolic-execution-example} shows the effectiveness of symbolic execution in identifying {\em all} the possible unsafe input values that can trigger a crash due to an unsafe division performed at line 10. This is achieved through an exhaustive exploration of all the possible execution states. For this reason, symbolic execution is a sound and complete methodology from a theoretical perspective. Soundness guarantees that input values deemed as unsafe are actually unsafe, and completeness implies that all possible unsafe inputs will be found. However, challenges that symbolic execution has to face when processing real-world code can be significantly more complex compared to those from our toy example of Figure~\ref{fig:example-1}. Several observations and questions naturally arise:

\begin{enumerate}

  \item (objects) {\em How does symbolic execution handle arrays or other more complex objects?} \\
  In general, any arbitrary complex object can be seen as an array of bytes, where each byte is associated with a distinct symbol. In principle, even a C {\tt int} variable can be seen as an array of four bytes.\mynote{Cosa intendiamo qui?} However, it is convenient to exploit structural properties of the data when possible (e.g., by statically analyzing the source code). For instance, for object-oriented languages the search performed by symbolic execution can be refined taking advantage of relational bounds on class fields.

  \item (loops) {\em How does symbolic execution handle loops?} \\
  In the execution model presented in Section~\ref{simple-execution-model} a loop can be encoded as a combination of conditional branches and $goto$ statements. This transformation is frequent when lowering a high-level language like C to an intermediate representation or native code. When the number of loop iterations cannot be determined in advance (e.g., it depends on an input parameter), for a symbolic execution engine choosing how many iterations should be analyzed becomes critical. The naive approach of unrolling iterations for every valid index bound leads to a very large number of states. It is possible to limit the number of iterations to $k$, thus trading speed for completeness, or when loop invariants can be inferred through static analysis they can be used to merge equivalent states (\mynote{Recuperare citazione}e.g., when differences are not observable outside the loop body).

  \item (subroutines) {\em How does symbolic execution handle subroutines?} \\
  Our execution model does not handle invocation of subroutines (i.e., a $call$ statement). \mynote{Extend this paragraph} A way to extend it to support subroutines is to provide the execution state with a simple execution stack.

  \item (recursion) {\em How does symbolic execution handle recursion?} \\
  Consider, as an example, the code:
    \begin{lstlisting}[basicstyle=\ttfamily\small]
    1.  int bar(int n) {
    2.    if (n >= 0) 
    3.      return 0;
    4.    return 1 + sum(n - 1);
    5.  }
    \end{lstlisting}
  Assuming that an execution stack has been added to the state, we observe that this code can easily lead to a very large number of execution states (i.e., a new state is subsequently created every time the branch on line 2 is not taken). As an {\tt int} variable can have up to  $2^{31} - 1$ positive values, symbolic execution has to create as many execution states to cover all the possible execution paths.
 %Indeed, the number of executions states is related to the number of times that the conditional branch on line 2 is not taken. 
 
  \item (environment) {\em How does symbolic execution handle interaction with the environment}? \\
  Real-world applications interacts constantly with the environment (e.g., filesystem, network) through libraries or system calls. A crucial aspect of these interactions is that they may cause side-effects
% on the environments 
(e.g., creation of a file)
%or initialization of a memory area
that must be taken into account, as they may later affect the 
%actual
execution of the code. Evaluating any possible outcome of an interaction is typically not feasible due to the large number of possibly generated execution states, only a small number of which can actually happen in a non-symbolic scenario. Hence, it is common to create models for popular library and system routines that help the symbolic execution engine to consider only significant outcomes.

  \item (state space explosion, path selection) {\em How does symbolic execution deal with path explosion}? \\
  A relatively simple code such as function {\tt foobar}, which is composed by less 12 lines of code, has generated 16 execution states, where $5$ out of $16$ are independent\mynote{Independent?} and must be checked to determine possible unsafe input values. Although this could seem a reasonable number of states, language constructs such as loops may contribute to increase the number of states exponentially. For this reason, it is unlikely that a symbolic execution engine is able to exhaustively explore all the possible execution states within a reasonable amount of time. In practice, heuristics are used to guide exploration and prioritize certain states first (e.g., to maximize code coverage), hoping this would lead to interesting discoveries. Also, a symbolic execution engine should implement efficient mechanism for evaluating multiple execution states in parallel without running out of resources.
  %In practice, several heuristics must be exploited to prioritize evaluation of some states, hoping to still be able to spot interesting things. Moreover, the symbolic execution engine should include efficient mechanism for efficiently evaluating in parallel different execution states without running out of computational resources.

  \item (constraint solver) {\em What can a constraint solver do in practice?}
  %{\em What is a constraint solver in practice}? \\
 Constraint solvers suffer from a number of limitations. Typically, they can handle complex constraints in a reasonable amount of time only if they are made of linear expressions over their constituents. Symbolic execution engines typically implement a number of optimizations to make queries as much {\em solver-friendly} as possible, for instance by splitting queries in independent components to process separately or by performing algebraic simplifications.

  \item (binary code) {\em What are the disadvantages of symbolically executing binary code}? \\
  The example presented in Section~\ref{symbolic-execution-example} is written in C. This does not imply that symbolic execution cannot be performed directly on binary code, which in several scenarios is the only available representation of a program. However, having the source code of an application makes symbolic execution significantly easier, as it can exploit high-level properties (e.g., object shapes) that can be inferred by statically analyzing the source code.
  %(e.g., the maximum size of a buffer or the number of iterations for a loop).
   
\end{enumerate}
%Depending on the specific application context of symbolic execution
Depending on the specific context in which symbolic execution is used, different choices and assumptions are made to address the questions highlighted above. Although they typically affect soundness or completeness, in several scenarios a partial exploration of the space of possible execution states is typically sufficient to reach the goal\mynote{Better example?} (e.g., identify a crashing input for an application) with a limited time budget.

%different choices and assumptions are made to address the above questions. Although soundness and completeness of symbolic execution may be negatively affected by these choices, there are several application scenarios where a partial exploration of the possible execution states is sufficient for reaching the ultimate goal (e.g., identify a single input that crashes an application).

\subsection{Paper organization}

%\vspace{2cm}
%\subsection{Removed stuff}
%
%\paragraph{Black-box approach versus white-box approach}
%
%Discussion\mynote{IF: do we really need this?} of black-box approach and white-box approach. Symbolic execution is a white-box technique. Black-box approaches can be very fast but not always effective. White-box approaches can be very effective but are typically slower than black-box techniques. An in-depth discussion of this aspect will be done when we will discuss~\cite{DRILLER-NDSS16}.
%
%\begin{figure}[H]
%  \vspace{-3mm}
%  \centering
%  \begin{subfigure}{.5\textwidth}
%    \centering
%    \includegraphics[width=0.9\linewidth]{images/blackbox} 
%    \caption{Black-box approach}
%    %\label{fig:sub1}
%  \end{subfigure}%
%  \begin{subfigure}{.5\textwidth}
%    \centering
%    \includegraphics[width=0.9\linewidth]{images/whitebox} 
%    \caption{White-box approach}
%    %\label{fig:sub2}
%  \end{subfigure}
%  %\label{fig:example-symbolic-execution}
%  \vspace{-3mm}
%\end{figure}
%
%\paragraph{Taken from old Overview}
%
%Symbolic execution has been originally introduced in~\cite{K-CACM76} and~\cite{H-TSE77}. A good introduction to symbolic execution is presented in~\cite{KLEE-OSDI08}.\mynote{Extend this paragraph}
%%(while~\cite{EXE-CCS06} is a previous effort of the same authors).
%\cite{SAGE-NDSS08} is one successful story of symbolic execution. \cite{SAB-SP10} presents a neat formalization of symbolic execution and of taint analysis as well.
%

\myinput{executors}
\myinput{memory}
\myinput{environment}
\myinput{explosion}
\myinput{constraints}
%\myinput{binary}
%% !TEX root = appendix.tex

\section{Sample Applications}
\label{se:applications}

%\revedit{
The last decade has witnessed an increasing adoption of symbolic execution techniques not only in the software testing domain, but also to address other compelling engineering problems such as automatic generation of exploits or authentication bypass. We now discuss \iffullver{three prominent}{prominent} applications of symbolic execution techniques to these domains. Examples of extensions to other areas can be found, e.g., in~\cite{CGK-ICSE11}.
%}

%The last decade has witnessed an increasing adoption of symbolic execution techniques not only in the software testing domain, but also to address other compelling engineering problems such as automatic generation of exploits or authentication bypass. We now discuss \iffullver{three prominent}{prominent} applications of symbolic execution techniques to these domains. Examples of extensions to other areas can be found, e.g., in~\cite{CGK-ICSE11}.

\subsection{Bug Detection}\mynote{Rendere piu' di ampio respiro il titolo di questa sezione? Keyword: software testing, program understanding}
\label{ss:bug-detection}

Software testing strategies typically attempt to execute a program with the intent of finding bugs. As manual test input generation is an error-prone and usually non-exhaustive process, automated testing technique have drawn a lot of attention over the years. Random testing techniques such as fuzzing are cheap in terms of run-time overhead, but fail to obtain a wide exploration of a program state space. Symbolic and concolic execution techniques on the other hand achieve a more exhaustive exploration, but they become expensive as the length of the execution grows: for this reason, they usually reveal shallow bugs only.

\cite{RK-ICSE07} proposes {\em hybrid concolic testing} for test input generation, which combines random search and concolic execution to achieve both deep program states and wide exploration. The two techniques are interleaved: in particular, when random testing saturates (i.e., it is unable to hit new code coverage points after a number of steps), concolic execution is used to mutate the current program state by performing a bounded depth-first search for an uncovered coverage point. For a fixed time budget, the technique outperforms both random and concolic testing in terms of branch coverage. The intuition behind this approach is that many programs show behaviors where a state can be easily reached through random testing, but then a precise sequence of events -- identifiable by a symbolic engine -- is required to hit a specific coverage point.

% which uses preconstraining on the program states to ensure consistency
\cite{DRILLER-NDSS16} refines this idea and devises a vulnerability excavation tool based on {\sc Angr}~\cite{ANGR-SSP16}, called Driller, that interleaves fuzzing and concolic execution to discover memory corruption vulnerabilities. The authors remark that user inputs can be categorized as {\em general} input, which has a wide range of valid values, and {\em specific} input: a check for particular values of a specific input then splits an application into {\em compartments}. Driller offloads the majority of unique path discovery to a fuzzy engine, and relies on concolic execution to move across compartments. During the fuzzy phase, Driller marks a number of inputs as interesting (for instance, when an input was the first to trigger some state transition) and once it gets stuck in the exploration, it passes the set of such paths to a concolic engine, which preconstraints the program states to ensure consistency with the results of the native execution. On the dataset used for the DARPA Cyber Grand Challenge qualifying event, Driller could identify crashing inputs in 77 applications, including both the 68 and 16 applications for which fuzzing and symbolic execution alone succeeded, respectively. For 6 applications, Driller was the only one to detect a vulnerability.

% temporaneamente messo qui
\mytempedit{
	Maintenance of large and complex applications is a very hard task. Fixing bugs can sometimes even introduce new and unexpected issues in the software, which in turn may require several hours or even weeks to be detected and properly addressed by the developers. \cite{QRL-TOSEM12} tackles the problem of identifying the root cause of failures during regression testing. Given a program $P$ and a newer revision of the program $P'$, if a testing input $t$ generates a failure in $P'$ but not in  $P$, then symbolic execution is used to track the path constraints $\pi$ and $\pi'$ when executing $P$ and $P'$ on the failing input $t$, respectively. Using a SMT solver, a new input $t'$ is generated by solving the constraint $\pi ~\wedge \neg\pi'$. If $t'$ exists (i.e., the constraint is satisfiable), then $P'$ has one or more {\em deviations} in the control-flow graph with respect to $P$ that can be the root cause of the failure. By carefully tracking branch conditions during symbolic execution, \cite{QRL-TOSEM12} are even able to pinpoint which branches are responsible for these deviations. If $\pi \wedge \neg\pi'$ is not satisfiable, then the symmetric constraint query $\neg\pi \wedge \pi'$ is tested and a similar reasoning is performed to detect the possible branch conditions that may have lad to the failure. If $\neg\pi \wedge \pi'$ is also unsatisfiable, then \cite{QRL-TOSEM12} cannot determine the root cause of the problem.

	Another interesting work that targets the problem of software regressions through the use of symbolic execution is~\cite{BOR-ICSE13}. This work introduce an approach called {\em partition-based regression verification} that combines the advantages of both regression verification (RV) and regression testing (RT). Indeed, RV is a very powerful technique for identifying regressions but hardly scales over large programs due to the difficulty in proving behavioral equivalence between the original and the modified program. On the other hand, RT allows to check a modified program for regressions by testing selected concrete sample inputs, making it more scalable but providing limited verification guarantees. The main intuition behind partition-based regression verification is to identify {\em differential partitions}. Each differential partition can be seen as a subset of the input space for which the two program versions -- given the same path constraints -- either expose the same output ({\em equivalence-revealing partition}) or produce different outputs ({\em difference-revealing partition}). For each partition, a test case is generated and added to the regression test suite, which can later be used by a developer for classical RT. Since differential partitions are derived exploiting symbolic execution, this approach suffers from the common limitations that come with this technique. However, if the exploration is interrupted (e.g., due to excessive time or memory usage), partition-based regression verification can still provide guarantees over the subset of input space that has been covered by the detected partitions.

	Static data flow analysis tools can significantly help developers tracks malicious data leaks in software applications. Unfortunately, they often report several allegedly bugs that only after manual inspection can be regarded as false positives. To mitigate this issue,~\cite{ARH-SOAP15} has proposed TASMAN, a system that, after performing data flow analysis to track information leaks, uses symbolic backward execution (SBE) to test each reported bug. Starting from a leaking statement, TASMAN backwards into the code, pruning any path that can be proved to be unfeasible. If all the paths starting from the leaking statement are discarded by TASMAN, then the reported bug can be marked as a false positive.

	Although symbolic execution has been extensively used for bug detection, during the last decades several works~\cite{GDV-ISSTA12,FPV-ICSE13,CLL-ICSE16} have shown how it can be also used for other program understanding activities. For instance,~\cite{GDV-ISSTA12} has introduced {\em probabilistic symbolic execution}, an approach that makes it possible to compute the probability of executing different code portions of a program. This is achieved by exploiting model counting techniques, such as the {\tt LattE}~\cite{LHT-JSC04} toolset, that allows~\cite{GDV-ISSTA12}  to determine the number of solutions for the different path constraints given by the alternative execution paths of a program. The paper by~\cite{FPV-ICSE13} makes a step further and uses probabilistic symbolic execution for performing software reliability analysis. This is computed as the probability of executing any path that has been labeled as successful given a usage profile. Intuitively, a usage profile can be seen as the distribution over the input space. Since in general the termination of symbolic execution cannot be guaranteed in presence of loops, then~\cite{FPV-ICSE13} resorts to bounded exploration. Nonetheless, they define a metric for evaluating the confidence in their reliability estimation, allowing a developer to increase the bounds in order to improve the confidence value. Of a different flavor is the work by~\cite{CLL-ICSE16} that exploits probabilistic symbolic execution to conduct performance analysis. Based on usage profiles and on path execution probabilities, paths are classified into two types: {\em low probability} and {\em high probability}. In a first phase, high-probability paths are explored in a way that maximizes path diversity, generating a first set of test inputs. In the second phase, low-probability paths are analyzed using symbolic execution, generating a second set of test inputs that should expose executions characterized by best-execution times and by worst-execution times. Finally, the program is executed using the test inputs generated during the two phases and running times are measured to generate performance distributions. 

	Another interesting application of symbolic execution is presented by~\cite{PPM-CSF18}. Their technique exploits model counting and symbolic execution for computing quantitative bounds on the amount of information that can be leaked by a program through side-channel attacks. 
	
}
%As it is based on {\sc Angr}, Driller adopts an index-based memory model as in Section~\ref{ss:index-based-memory} where reads can be symbolic and writes are always concretized. % read/write addresses

\subsection{Bug Exploitation}
\label{ss:bug-exploitation}
Bugs are a consequence of the nature of human factors in software development and are everywhere. Those that can be exploited by an attacker should normally be fixed first: systems for automatically and effectively identifying them are thus very valuable.

{\sc AEG}~\cite{AEG-NDSS11} employs preconditioned symbolic execution to analyze a potentially buggy program in source form and look for bugs amenable to stack smashing or return-into-libc exploits~\cite{PB-SSP04}, which are popular control hijack attack techniques. The tool augments path constraints with exploitability constraints and queries a constraint solver, generating a concrete exploit when the constraints are satisfiable. The authors devise the {\em buggy-path-first} and {\em loop-exhaustion} strategies (Table~\ref{tab:heuristics}) to prioritize paths in the search. On a suite of 14 Linux applications, {\sc AEG} discovered 16 vulnerabilities, 2 of which were previously unknown, and constructed control hijack exploits for them.

{\sc Mayhem}~\cite{MAYHEM-SP12} takes another step forward by presenting the first system for binary programs that is able identify end-to-end exploitable bugs. It adopts a hybrid execution model based on checkpoints and two components: a concrete executor that injects taint-analysis instrumentation in the code and a symbolic executor that takes over when a tainted branch or jump instruction is met. Exploitability constraints for symbolic instruction pointers and format strings are generated, targeting a wide range of exploits, e.g., SEH-based and jump-to-register ones. Three path selection heuristics help prioritizing paths that are most likely to contain vulnerabilities (e.g., those containing symbolic memory accesses or instruction pointers). A virtualization layer intercepts and emulates all the system calls to the host OS, while preconditioned symbolic execution can be used to reduce the size of the search space. Also, restricting symbolic execution to tainted basic blocks only gives very good speedups in this setting, as in the reported experiments more than $95\%$ of the processed instructions were not tainted. {\sc Mayhem} was able to find exploitable vulnerabilities in the 29 Linux and Windows applications considered in the evaluation, 2 of which were previously undocumented. Although the goal in {\sc Mayhem} is to reveal exploitable bugs, the generated simple exploits can be likely transformed in an automated fashion to work in the presence of classical OS defenses such as data execution prevention and address space layout randomization~\cite{Q-SEC11}. 

\vspace{-1mm} % TODO
\subsection{Authentication Bypass}
\label{ss:auth-bypass}
Software backdoors are a method of bypassing authentication in an algorithm, a software product, or even in a full computer system. Although sometimes these software flaws are injected by external attackers using subtle tricks such as compiler tampering~\cite{KRS-TR74}, there are reported cases of backdoors that have been surreptitiously installed by the hardware and/or software manufacturers~\cite{CZF-USEC14}, or even by governments~\cite{NSA-BACKDOOR}. 

Different works~\cite{DMR-USEC13,ZBF-NDSS14,FIRMALICE-NDSS15} have exploited symbolic execution for analyzing the behavior of binary firmwares. Indeed, an advantage of this technique is that it can be used even in environments, such as embedded systems, where the documentation and the source code that are publicly released by the manufacturer are typically very limited or none at all. For instance,~\cite{FIRMALICE-NDSS15} proposes Firmalice, a binary analysis framework based on {\sc Angr}~\cite{ANGR-SSP16} that can be effectively used for identifying authentication bypass flaws inside firmwares running on devices such as routers and printers. Given a user-provided description of a privileged operation in the device, Firmalice identifies a set of program points that, if executed, forces the privileged operation to be performed. The program slice that involves the privileged program points is then symbolically analyzed using {\sc Angr}. If any such point can be reached by the engine, a set of concrete inputs is generated using an SMT solver. These values can be then used to effectively bypass authentication inside the device. On three commercially available devices, Firmalice could detect vulnerabilities in two of them, and determine that a backdoor in the third firmware is not remotely exploitable.
% !TEX root = main.tex

\section{Further Directions}
\label{se:hang}
\mytempedit{
In this section we \myedit{discuss} how recent advances in related research areas could be applied or provide potential directions to enhance the state of the art of symbolic execution techniques. In particular, we discuss separation logic for data structures, techniques from the program verification and program analysis domains for dealing with path explosion, and symbolic computation for dealing with non-linear constraints.
%}
% , i.e., polynomial constraints over variables

% Reynolds02,IO-POPL01
\subsection{Separation Logic}
%\mytempedit{
Checking memory safety properties for pointer programs is a major challenge in program verification. Recent years have witnessed {\em separation logic} (SL)~\cite{Reynolds02} emerging as one leading approach to reason about heap manipulations in imperative programs. SL extends Hoare logic to facilitate reasoning about programs that manipulate pointer data structures, and allows expressing complex invariants of heap configurations in a succinct manner.

At its core, a {\em separating conjunction} binary operator $*$ is used to assert that the heap can be partitioned in two components where its arguments separately hold. For instance, predicate $A * x\mapsto�[n:y]$ says that there is a single heap cell $x$ pointing to a record that holds $y$ in its $n$ field, while $A$ holds for the rest of the heap.

Program state is modeled as a {\em symbolic heap} $\Pi\,\brokenvert\,\Sigma$: $\Pi$ is a finite set of pure predicates related to variables, while $\Sigma$ is a finite set of heap predicates. Symbolic heaps are SL formulas that are symbolically executed according to the program's code using an abstract semantics. SL rules are typically employed to support entailment of symbolic heaps, to infer which heap portions are not affected by a statement, and to ensure termination of symbolic execution via abstraction (e.g., using a widening operator).

A key to the success of SL lies in the local form of reasoning enabled by its $*$ operator, as it allows specifications that speak about the sole memory accessed by the code. This also fits together with the goal of deriving inductive definitions to describe mutable data structures. When compared to other verification approaches, the annotation burden on the user is rather little or often absent. For instance, the shape analysis presented in~\cite{CDO-JACM11} uses bi-abduction to automatically discover invariants on data structures and compute composable procedure summaries in SL.

% verification (Section~\ref{se:constraint-solving})
Several tools based on SL are available to date, for instance, for automatic memory bug discovery in user and system code, and verification of annotated programs against memory safety properties or design patterns. While some of them implement tailor-made decision procedures, \cite{BPS-ENTCS09,PWZ-CAV13} have shown that provers for decidable SL fragments can be integrated in an SMT solver, allowing for complete combinations with other theories relevant to program verification. This can pave the way to applications of SL in a broader setting: for instance, a symbolic executor could use it to reason inductively about code that manipulates structures such as lists and trees. While symbolic execution is at the core of SL, to the best of our knowledge there have not been uses of SL in symbolic executors to date.%We believe this might represent a promising research direction to follow.
%}

\subsection{Invariants} 
\label{ss:invariants}
%\mytempedit{
% is not just a quantity that remains unchanged throughout executions of the loop body, but
Loop invariants play a key role in verifiers that can prove programs correct against their full functional specification. An invariant is an inductive property that holds when the loop is first entered and is preserved for an arbitrary number of iterations~\cite{FMV-CSUR14,GFM-TSE15}. Leveraging invariants can be beneficial to symbolic executors, in order to compactly capture the effects of a loop and reason about them. Unfortunately, we are not aware of symbolic executors taking advantage of this approach. One of the reasons might lie in the difficulty of computing loop invariants without requiring manual intervention from domain experts. In fact, lessons from the verification practice suggest that providing loop invariants is much harder compared to other specification elements such as method pre/post-conditions.

% a number of works [...] and might be
However, many researchers have recently explored techniques for inferring loop invariants automatically or with little human help~\cite{FMV-CSUR14}, which might be of interest for the symbolic execution community for a more efficient handling of loops.

% that rank all -> over all
{\em Termination analysis} has been applied to verify program termination for industrial code: a formal argument is typically built by using one or more ranking functions over all the possible states in the program such that for every state transition, at least one function decreases~\cite{CPR-PLDI06}. Ranking functions can be constructed in a number of ways, e.g., by lazily building an invariant using counterexamples from traversed loop paths~\cite{GMR-PLDI15}. A termination argument can also be built by reasoning over transformed programs where loops are replaced with summaries based on transition invariants~\cite{TSW-TACAS11}. It has been observed that most loops in practice have relatively simple termination arguments~\cite{TSW-TACAS11}: the discovered invariants may thus not be rich enough for a verification setting~\cite{GFM-TSE15}. However, a constant or parametric bound on the number of iterations may still be computed from a ranking function and an invariant~\cite{GMR-PLDI15}.

{\em Predicate abstraction} is a form of abstract interpretation over a domain constructed using a given set of predicates, and has been used to infer universally quantified loop invariants~\cite{FQ-POPL02}, which are useful when manipulating arrays. Predicates can be heuristically collected from the code or supplied by the user: it would be interesting to explore a mutual reinforcing combination with symbolic execution, with additional useful predicates being originated during the symbolic exploration.

{\em LoopFrog}~\cite{LOOPFROG-ATVA08} replaces loops using a symbolic abstract transformer with respect to a set of abstract domains, obtaining a conservative abstraction of the original code. Abstract transformers are computed starting from the innermost loop, and the output is a loop-free summary of the program that can be handed to a model checker for verification. This approach can also be applied to non-recursive function calls, and might deserve some investigation in symbolic executors. 

Loop invariants can also be extracted using {\em interpolation}, a general technique that has already been applied in symbolic execution for different goals (Section~\ref{ss:interpolation}). %More generally, we believe that modern advances in invariant generation can provide potential solutions for handling loops more efficiently in a symbolic executor.
%
%}

% DROPPED as it's an application
\iffullver{
On the other hand, symbolic execution has proven useful to derive loop invariants. For instance, if a program contains an assertion after the loop, the approach presented in~\cite{PV-SPIN04} works backwards from the property to be checked and it iteratively applies approximation to derive loop invariants. The main idea is to pick the asserted property as the initial invariant candidate and then to exploit symbolic execution to check whether this property is inductive. If the invariant cannot be verified for some loop paths, it is replaced by a different invariant. The next candidate for the invariant is generated by exploiting the path constraints for the paths on which the verification has failed. Additional refinements steps are performed to guarantee termination. % [D] say weakness in not being able to find invariants that do not directly depends on the path conditions?
}{}

%Nevertheless, even symbolic execution can be used to derive loop invariants. Indeed, if a program contains an assertion after the loop, the approach presented in~\cite{PV-SPIN04} works backwards from the property to be checked and it iteratively applies approximation to derive loop invariants. The main idea is to pick the asserted property as the initial invariant candidate and then to exploit symbolic execution to check whether this property is inductive. If the invariant cannot be verified for some loop paths, it is replaced by a different invariant. The next candidate for the invariant is generated by exploiting the path constraints for the paths on which the verification has failed. Additional refinements steps are performed to guarantee termination.
%this can be exploited by a symbolic engine for automatically discovering some invariants over the loop. In~\cite{PV-SPIN04}, this is achieved by iteratively using \mynote{[D] Define?} invariant strengthening and approximation techniques. 

\subsection{Function Summaries}
%\mytempedit{
Function summaries (Section~\ref{ss:summarization}) have largely been employed in static and dynamic program analysis, especially in program verification. A number of such works could offer interesting opportunities to advance the state of the art in symbolic execution. For instance, the Calysto static checker~\cite{CALYSTO-ICSE08} walks the call graph of a program to construct a symbolic representation of the effects of each function, i.e., return values, writes to global variables, and memory locations accessed depending on its arguments. Each function is processed once, possibly inlining effects of small ones at their call sites. Static checkers such as Calysto and Saturn~\cite{SATURN-POPL05} trade scalability for soundness in summary construction, as they unroll loops only to a small number of iterations: their use in a symbolic execution setting may thus result in a loss of soundness. More fine-grained summaries are constructed in~\cite{RACERX-SOSP03} by taking into account different input conditions using a summary cache for memoizing the effects of a function.

\cite{SFS11} proposes a technique to extract function summaries for model checking where multiple specifications are typically checked one a time, so that summaries can be reused across verification runs. In particular, they are computed as over-approximations using interpolation (Section~\ref{ss:interpolation}) and refined across runs when too weak. The strength of this technique lies in the fact that an interpolant-based summary can capture all the possible execution traces through a function in a more compact way than the function itself. The technique has later been extended to deal with nested function calls in~\cite{SFS12}.%, which discusses an useful application in incremental update checking of programs.
%}

\subsection{Program analysis and optimization}
%\mytempedit{
We believe that the symbolic execution practice might also benefit from solutions that have been proposed for related problems in the program analysis realm. For instance, in the parallel computing community transformations such as {\em loop coalescing}~\cite{BGS-CSUR94} can restructure nested loops into a single loop by flattening the iteration space of their indexes. Such a transformation could potentially simplify a symbolic exploration, empowering search heuristics and state merging strategies. 

{\em Loop unfolding}~\cite{SK-SIGPLAN-NOTICES04} may potentially be interesting as well, as it allows exposing ``well-structured'' loops (e.g., showing invariant code, or having constants or affine functions as subscripts of array references) by peeling several iterations.

{\em Program synthesis}~\cite{PR-POPL89} automatically constructs a program satisfying a high-level specification. The technique has caught the attention of the verification community since~\cite{Solar-Lezama08} has shown how to find programs as a solution to SAT problems.
In Section~\ref{se:environment-thirdparty} we discussed its usage in~\cite{JQF-ICSE16} to produce compact models for complex Java frameworks: the technique takes as inputs classes, methods and types from a framework, along with tutorial programs (typically those provided by the vendor) that exercise its parts. We believe this approach deserves further investigation in the context of the path explosion problem. It could potentially be applied to software modules such as standard libraries to produce concise models that allow for a more scalable exploration of the search space, as synthesis can capture an external behavior while abstracting away internal implementation entanglements.
%has shown that program synthesis can be used to build models for complex Java components by abstracting away the details and entanglements of their implementations while capturing their functional behavior. In particular, 

\subsection{Symbolic Computation}
%In particular, studies in the area of {\em symbolic computation}, also known as computer algebra,
Although the satisfiability problem is known to be NP-hard already for SAT, the mathematical developments over the past decades have produced several practically applicable methods to solve arithmetic formulas. In particular, advances in {\em symbolic computation} have produced powerful methods such as Gr\"{o}bner bases for solving systems of polynomial constraints, cylindrical algebraic decomposition for real algebraic geometry, and virtual substitution for non-linear real arithmetic formulas~\cite{Abraham15}.

% handling complex Boolean constraints and; quantifier-free non-linear
% have proven to be
While SMT solvers are very efficient at combining theories and heuristics when processing complex expressions, they make use of symbolic computation techniques only to a little extent, and their support for non-linear real and integer arithmetic is still in its infancy~\cite{Abraham15}. To the best of our knowledge, only Z3~\cite{Z3-TACS08} and SMT-RAT~\cite{SMTRAT15} can reason about them both.
%provide support to reason about them both.

\cite{Abraham15} states that using symbolic computation techniques as theory plugins for SMT solvers is a promising symbiosis, as they provide powerful procedures for solving conjunctions of arithmetic constraints. The realisation of this idea is hindered by the fact that available implementations of such procedures do not comply with the incremental, backtracking and explanation of inconsistencies properties expected of SMT-compliant theory solvers. One interesting project to look at is SC\textsuperscript{2}~\cite{SC2}, whose goal is to create a new community aiming at bridging the gap between symbolic computation and satisfiability checking, combining the strengths of both worlds in order to pursue problems currently beyond their individual reach.

Further opportunities to increase efficiency when tackling non-linear expressions might be found in the recent advances in {\em symbolic-numeric computation}~\cite{HandbookOfCompAlgebra}. In particular, these techniques aim at developing efficient polynomial solvers by combining numerical algorithms, which are very efficient in approximating local solutions but lack a global view, with the guarantees from symbolic computation techniques. This hybrid techniques can extend the domain of efficiently solvable problems, and thus be of interest for non-linear constraints from symbolic execution.
}

% !TEX root = paper.tex
\section{Conclusions}
\label{se:conclusions}

Techniques for symbolic execution have evolved significantly in the last decade, leading to major practical breakthroughs. In 2016, the DARPA Cyber Grand Challenge hosted systems that can detect and fix vulnerabilities in unknown software with no human intervention, such as {\sc Angr}~\cite{ANGR-SSP16} and {\sc Mayhem}~\cite{MAYHEM-SP12}, which won the \$2M first prize. {\sc Mayhem} was also the first autonomous software to play the Capture-The-Flag contest at the DEF CON 24 hacker convention. The event demonstrated that tools for automatic exploit detection based on symbolic execution can be competitive with human experts, paving the road to unprecedented applications and the rise of start-ups that have the potential to shape software security and reliability in the next decades. 

For details on the techniques behind symbolic executors, we would like to suggest the interested reader to have a look at a survey we have recently completed. The survey, available at \url{https://arxiv.org/abs/1610.00502}, has been written with the goal of providing an overview of the main ideas, challenges, and solutions developed in the area, distilling them for a broad audience.

\iffalse

Techniques for symbolic execution have evolved significantly in the last decade, leading to major practical breakthroughs. In 2016, the DARPA Cyber Grand Challenge hosted systems that can detect and fix vulnerabilities in unknown software with no human intervention, such as {\sc Angr}~\cite{ANGR-SSP16} and {\sc Mayhem}~\cite{MAYHEM-SP12}, which won the \$2M first prize. {\sc Mayhem} was also the first autonomous software to play the Capture-The-Flag contest at the DEF CON 24 hacker convention\footnote{\url{https://www.defcon.org/html/defcon-24/dc-24-ctf.html}.}. The event demonstrated that tools for automatic exploit detection based on symbolic execution can be competitive with human experts, paving the road to unprecedented applications %and the rise of start-ups 
that have the potential to shape software %security and 
reliability in the next decades. 

This survey has discussed some of the key aspects and challenges of symbolic execution, presenting them for a broad audience. To explain the basic design principles of symbolic executors and the main optimization techniques, we have focused on single-threaded applications with integer arithmetic. Symbolic execution of multi-threaded programs is treated, e.g., \iffullver{in~\cite{KPV-TACAS03,SA-HVC06,CLOUD9-EUROSYS11,FHR-ESEC13,BGC-OOPSLA14,GKW-ESEC15}}{in~\cite{FHR-ESEC13,BGC-OOPSLA14,GKW-ESEC15}}, while techniques for programs that manipulate floating point data are addressed \iffullver{in, e.g., \cite{M-STVR01,BGM-STVR06,LTH-ICTSS10,CCK-EUROSYS11,BVL-POPL13,CCK-TSE14,RPW-SIGSOFT15}}{in, e.g., \cite{BVL-POPL13,CCK-TSE14,RPW-SIGSOFT15}}.

We hope that this survey will help non-experts grasp the key inventions in the exciting line of research of symbolic execution, inspiring further work and new ideas.

\ifdefined\arxivver

\myparagraph{Acknowledgements}
This work is supported in part by a grant of the Italian Presidency of the Council of Ministers and by the CINI (Consorzio Interuniversitario Nazionale Informatica) National Laboratory of Cyber Security.

\myparagraph{Live Version of this Article}
We complement the traditional scholarly publication model by maintaining a live version of this article at {\href{https://github.com/season-lab/survey-symbolic-execution}{https://github.com/season-lab/survey-symbolic-execution/}}. The live version incorporates continuous feedback by the community, providing post-publication fixes, improvements, and extensions.
\fi

\fi


\myparagraph{Acknowledgements}
%This work is partially supported by a grant of the Italian Presidency of Ministry Council and by the CINI  (Consorzio Interuniversitario Nazionale Informatica) Cybersecurity National Laboratory.
This work is supported in part by a grant of the Italian Presidency of the Council of Ministers and by the CINI (Consorzio Interuniversitario Nazionale Informatica) National Laboratory of Cyber Security.


% !TEX root = main.tex

\section{Glossary}
\label{se:glossary}

\noindent {\bf Complete analysis.} Analysis that guarantees no false positives, i.e., all reported property violations are true.

\smallskip\noindent {\bf Concrete execution.} An execution of a program using concrete inputs in a real-world environment.

\smallskip\noindent {\bf Concolic execution.} \ldots

\smallskip\noindent {\bf Control flow graph (CFG).} Representation of a program that uses nodes to model instructions and edges to model the control flow between them.

\smallskip\noindent {\bf Control flow path.} Path in the control flow graph of a program. Represents the sequence of instructions executed by the program for a given concrete input.

\smallskip\noindent {\bf Decidable analysis} \ldots

\smallskip\noindent {\bf Model checker.} Given a model of a system, a model checker exhaustively and automatically checks whether the model meets a given specification.

\smallskip\noindent {\bf Symbolic execution.} \ldots

\smallskip\noindent {\bf Symbolic store.} \ldots

\smallskip\noindent {\bf SMT solver.} A Satisfiability Modulo Theories (SMT) instance is a formula in first-order logic, where some function and predicate symbols have additional interpretations, and SMT is the problem of determining whether such a formula is satisfiable. A SMT solver is a tool able to reason over SMT formulas.

\smallskip\noindent {\bf Sound analysis.} Analysis that guarantees no false negatives, i.e., if there is a property violation, then it is reported.

\smallskip\noindent {\bf Path constraints.} \ldots

\appendix
% !TEX root = appendix.tex

\section{Additional Tables}

\begin{table}[b]
  \centering
  %\begin{adjustbox}{width=\columnwidth}
  \begin{small}
  \begin{tabular}{| l || c || l |}
    \hline      
    {\bf Symbolic engine} & {\bf References} & {\bf Project URL} (last retrieved: December 2017)  \\ \hline\hline
    
    % CNC is not a symbolic engine but it uses constrained solver
    %{\sc Check 'n' Crash} & \cite{CS-ICSE05} & \url{http://ranger.uta.edu/~csallner/cnc/}\\
    
    {\sc CUTE} & \cite{CUTE-FSE05} & -- \\
    {\sc DART} & \cite{DART-PLDI05} & -- \\
    {\sc jCUTE} & \cite{SA-CAV06} & \url{https://github.com/osl/jcute} \\ % : Java Concolic Unit Testing Engine
    {\sc KLEE} & \cite{EXE-CCS06,KLEE-OSDI08} & \url{https://klee.github.io/} \\ % : a LLVM Execution Engine
    {\sc SAGE} & \cite{SAGE-NDSS08,EGL-ISSTA09} & -- \\
    {\sc BitBlaze} & \cite{BITBLAZE-ICISS08} & \url{http://bitblaze.cs.berkeley.edu/} \\ % , BHK-TR07
    {\sc CREST} & \cite{CREST-ASE08} & \url{https://github.com/jburnim/crest} \\ % : a concolic test generation tool for C
    {\sc PEX} & \cite{PEX-TAP08} & \url{http://research.microsoft.com/en-us/projects/pex/} \\
    {\sc Rubyx} & \cite{CF-CCS10} & -- \\
    {\sc Java PathFinder} & \cite{PATHFINDER-ASE10} & \url{http://babelfish.arc.nasa.gov/trac/jpf}\\
    {\sc Otter} & \cite{RSM-ICSE10} & \url{https://bitbucket.org/khooyp/otter/} \\
    {\sc BAP} & \cite{BAP-CAV11} & \url{https://github.com/BinaryAnalysisPlatform/bap} \\
    {\sc Cloud9} & \cite{CLOUD9-EUROSYS11} & \url{http://cloud9.epfl.ch/} \\
    {\sc Mayhem} & \cite{MAYHEM-SP12} & -- \\
    {\sc SymDroid} & \cite{JMF-TECH12} & -- \\
    {\sc \stwoe} & \cite{CKC-TOCS12} & \url{http://s2e.systems/} \\
    {\sc FuzzBALL} & \cite{MMP-ASPLOS12,FUZZBALL-ESORICS13} & \url{http://bitblaze.cs.berkeley.edu/fuzzball.html} \\
    {\sc Jalangi} & \cite{SKB-FSE13} & \url{https://github.com/Samsung/jalangi2} \\
    {\sc Pathgrind} & \cite{S-ICSE04} & \url{https://github.com/codelion/pathgrind} \\
    {\sc Kite} & \cite{V-THESIS14} & \url{http://www.cs.ubc.ca/labs/isd/Projects/Kite} \\
    {\sc SymJS} & \cite{LAG-FSE14} & -- \\
    {\sc CIVL} & \cite{CIVL-SC15} & \url{http://vsl.cis.udel.edu/civl/}\\ % : The Concurrency Intermediate Verification Language 
    {\sc KeY} & \cite{HBR-RV14} & \url{http://www.key-project.org/} \\
    {\sc Angr} & \cite{FIRMALICE-NDSS15,ANGR-SSP16} & \url{http://angr.io/} \\
    {\sc Triton} & \cite{TRITON-SSTIC15} & \url{http://triton.quarkslab.com/} \\
    {\sc PyExZ3} & \cite{BD-TECH15} & \url{https://github.com/thomasjball/PyExZ3} \\
    {\sc JDart} & \cite{JDART-TACAS16} & \url{https://github.com/psycopaths/jdart} \\

    {\sc CATG} & -- & \url{https://github.com/ksen007/janala2} \\
    {\sc PySymEmu} & -- & \url{https://github.com/feliam/pysymemu/} \\
    {\sc Miasm} & -- & \url{https://github.com/cea-sec/miasm} \\
    
    \hline  
  \end{tabular}
  \end{small}
  %\end{adjustbox}
  \caption{Selection of symbolic execution engines, along with their reference article(s) and software project web site (if any).}
  \label{tab:symbolic-engines}
  \vspace{-3.2mm} % TODO
\end{table}

\vspace{-2pt}
\myparagraph{Tools}
Table~\ref{tab:symbolic-engines} lists a number of symbolic execution engines that have worked as incubators for several of the techniques surveyed in this article. The novel contributions introduced by tools that represented milestones in the area are described in the appropriate sections throughout the main article.

\vspace{-1pt}
\myparagraph{Path Selection Heuristics}
Table~\ref{tab:heuristics} provides a categorization of the search heuristics that have been discussed in Section 2.3 of the main article. For each category, we list several works that have proposed interesting embodiments of the category.

\begin{table}[t]
  \centering
  \begin{adjustbox}{width=0.95\linewidth} % TODO was 1; with 0.88 the last paragraph will fit
  \begin{small}
  \begin{tabular}{| l || l |}
    \hline      
    {\bf Heuristic} & {\bf Goal} \\ \hline\hline
    BFS & {\em Maximize coverage} \cite{CKC-TOCS12,PEX-TAP08} \\ \hline
    DFS & {\em Exhaust paths, minimize memory usage} \cite{EXE-CCS06,CKC-TOCS12,PEX-TAP08,DART-PLDI05} \\\hline
    Random path selection & {\em Randomly pick a path with probability based on its length} \cite{KLEE-OSDI08} \\\hline
    %low-covered code & prioritize paths that execute low-covered code  & \cite{EXE-CCS06} \\
    \multirow{2}*{Code coverage search} & {\em Prioritize paths that may explore unexplored code or that may soon} \\ & {\em reach a particular target program point}  \cite{EXE-CCS06,KLEE-OSDI08,MAYHEM-SP12,CKC-TOCS12,GV-ISSTA02,MPF-SAS11} \\\hline
    Buggy-path-first & {\em Prioritize bug-friendly path} \cite{AEG-NDSS11} \\\hline
    Loop exhaustion & {\em Fully explore specific loops} \cite{AEG-NDSS11} \\\hline
    Symbolic instruction pointers & {\em Prioritize paths with symbolic instruction pointers} \cite{MAYHEM-SP12} \\\hline
    Symbolic memory accesses & {\em Prioritize paths with symbolic memory accesses} \cite{MAYHEM-SP12} \\ \hline
    Fitness function & {\em Prioritize paths based on a fitness function} \cite{XTD-DSN09,CS-CACM13,XTD-DSN09} \\ \hline
    \multirow{2}*{Subpath-guided search} & {\em Use frequency distributions of explored subpaths to prioritize less covered}\\ & {\em parts of a program} \cite{LZL-OOPSLA13} \\ \hline
    Property-guided search & {\em Prioritize paths that are most likely to satisfy the target property} \cite{ZCWDL15} \\ 
    %kill path & filter uninteresting path & \cite{CKC-TOCS12} \\
    \hline  
  \end{tabular}
  \end{small}
  \end{adjustbox}
  \caption{Common path selection heuristics discussed in Section 2.3.} % of the main article
  \label{tab:heuristics}
\end{table}

\myinput{binary}
% !TEX root = appendix.tex

\section{Sample Applications}
\label{se:applications}

%\revedit{
The last decade has witnessed an increasing adoption of symbolic execution techniques not only in the software testing domain, but also to address other compelling engineering problems such as automatic generation of exploits or authentication bypass. We now discuss \iffullver{three prominent}{prominent} applications of symbolic execution techniques to these domains. Examples of extensions to other areas can be found, e.g., in~\cite{CGK-ICSE11}.
%}

%The last decade has witnessed an increasing adoption of symbolic execution techniques not only in the software testing domain, but also to address other compelling engineering problems such as automatic generation of exploits or authentication bypass. We now discuss \iffullver{three prominent}{prominent} applications of symbolic execution techniques to these domains. Examples of extensions to other areas can be found, e.g., in~\cite{CGK-ICSE11}.

\subsection{Bug Detection}\mynote{Rendere piu' di ampio respiro il titolo di questa sezione? Keyword: software testing, program understanding}
\label{ss:bug-detection}

Software testing strategies typically attempt to execute a program with the intent of finding bugs. As manual test input generation is an error-prone and usually non-exhaustive process, automated testing technique have drawn a lot of attention over the years. Random testing techniques such as fuzzing are cheap in terms of run-time overhead, but fail to obtain a wide exploration of a program state space. Symbolic and concolic execution techniques on the other hand achieve a more exhaustive exploration, but they become expensive as the length of the execution grows: for this reason, they usually reveal shallow bugs only.

\cite{RK-ICSE07} proposes {\em hybrid concolic testing} for test input generation, which combines random search and concolic execution to achieve both deep program states and wide exploration. The two techniques are interleaved: in particular, when random testing saturates (i.e., it is unable to hit new code coverage points after a number of steps), concolic execution is used to mutate the current program state by performing a bounded depth-first search for an uncovered coverage point. For a fixed time budget, the technique outperforms both random and concolic testing in terms of branch coverage. The intuition behind this approach is that many programs show behaviors where a state can be easily reached through random testing, but then a precise sequence of events -- identifiable by a symbolic engine -- is required to hit a specific coverage point.

% which uses preconstraining on the program states to ensure consistency
\cite{DRILLER-NDSS16} refines this idea and devises a vulnerability excavation tool based on {\sc Angr}~\cite{ANGR-SSP16}, called Driller, that interleaves fuzzing and concolic execution to discover memory corruption vulnerabilities. The authors remark that user inputs can be categorized as {\em general} input, which has a wide range of valid values, and {\em specific} input: a check for particular values of a specific input then splits an application into {\em compartments}. Driller offloads the majority of unique path discovery to a fuzzy engine, and relies on concolic execution to move across compartments. During the fuzzy phase, Driller marks a number of inputs as interesting (for instance, when an input was the first to trigger some state transition) and once it gets stuck in the exploration, it passes the set of such paths to a concolic engine, which preconstraints the program states to ensure consistency with the results of the native execution. On the dataset used for the DARPA Cyber Grand Challenge qualifying event, Driller could identify crashing inputs in 77 applications, including both the 68 and 16 applications for which fuzzing and symbolic execution alone succeeded, respectively. For 6 applications, Driller was the only one to detect a vulnerability.

% temporaneamente messo qui
\mytempedit{
	Maintenance of large and complex applications is a very hard task. Fixing bugs can sometimes even introduce new and unexpected issues in the software, which in turn may require several hours or even weeks to be detected and properly addressed by the developers. \cite{QRL-TOSEM12} tackles the problem of identifying the root cause of failures during regression testing. Given a program $P$ and a newer revision of the program $P'$, if a testing input $t$ generates a failure in $P'$ but not in  $P$, then symbolic execution is used to track the path constraints $\pi$ and $\pi'$ when executing $P$ and $P'$ on the failing input $t$, respectively. Using a SMT solver, a new input $t'$ is generated by solving the constraint $\pi ~\wedge \neg\pi'$. If $t'$ exists (i.e., the constraint is satisfiable), then $P'$ has one or more {\em deviations} in the control-flow graph with respect to $P$ that can be the root cause of the failure. By carefully tracking branch conditions during symbolic execution, \cite{QRL-TOSEM12} are even able to pinpoint which branches are responsible for these deviations. If $\pi \wedge \neg\pi'$ is not satisfiable, then the symmetric constraint query $\neg\pi \wedge \pi'$ is tested and a similar reasoning is performed to detect the possible branch conditions that may have lad to the failure. If $\neg\pi \wedge \pi'$ is also unsatisfiable, then \cite{QRL-TOSEM12} cannot determine the root cause of the problem.

	Another interesting work that targets the problem of software regressions through the use of symbolic execution is~\cite{BOR-ICSE13}. This work introduce an approach called {\em partition-based regression verification} that combines the advantages of both regression verification (RV) and regression testing (RT). Indeed, RV is a very powerful technique for identifying regressions but hardly scales over large programs due to the difficulty in proving behavioral equivalence between the original and the modified program. On the other hand, RT allows to check a modified program for regressions by testing selected concrete sample inputs, making it more scalable but providing limited verification guarantees. The main intuition behind partition-based regression verification is to identify {\em differential partitions}. Each differential partition can be seen as a subset of the input space for which the two program versions -- given the same path constraints -- either expose the same output ({\em equivalence-revealing partition}) or produce different outputs ({\em difference-revealing partition}). For each partition, a test case is generated and added to the regression test suite, which can later be used by a developer for classical RT. Since differential partitions are derived exploiting symbolic execution, this approach suffers from the common limitations that come with this technique. However, if the exploration is interrupted (e.g., due to excessive time or memory usage), partition-based regression verification can still provide guarantees over the subset of input space that has been covered by the detected partitions.

	Static data flow analysis tools can significantly help developers tracks malicious data leaks in software applications. Unfortunately, they often report several allegedly bugs that only after manual inspection can be regarded as false positives. To mitigate this issue,~\cite{ARH-SOAP15} has proposed TASMAN, a system that, after performing data flow analysis to track information leaks, uses symbolic backward execution (SBE) to test each reported bug. Starting from a leaking statement, TASMAN backwards into the code, pruning any path that can be proved to be unfeasible. If all the paths starting from the leaking statement are discarded by TASMAN, then the reported bug can be marked as a false positive.

	Although symbolic execution has been extensively used for bug detection, during the last decades several works~\cite{GDV-ISSTA12,FPV-ICSE13,CLL-ICSE16} have shown how it can be also used for other program understanding activities. For instance,~\cite{GDV-ISSTA12} has introduced {\em probabilistic symbolic execution}, an approach that makes it possible to compute the probability of executing different code portions of a program. This is achieved by exploiting model counting techniques, such as the {\tt LattE}~\cite{LHT-JSC04} toolset, that allows~\cite{GDV-ISSTA12}  to determine the number of solutions for the different path constraints given by the alternative execution paths of a program. The paper by~\cite{FPV-ICSE13} makes a step further and uses probabilistic symbolic execution for performing software reliability analysis. This is computed as the probability of executing any path that has been labeled as successful given a usage profile. Intuitively, a usage profile can be seen as the distribution over the input space. Since in general the termination of symbolic execution cannot be guaranteed in presence of loops, then~\cite{FPV-ICSE13} resorts to bounded exploration. Nonetheless, they define a metric for evaluating the confidence in their reliability estimation, allowing a developer to increase the bounds in order to improve the confidence value. Of a different flavor is the work by~\cite{CLL-ICSE16} that exploits probabilistic symbolic execution to conduct performance analysis. Based on usage profiles and on path execution probabilities, paths are classified into two types: {\em low probability} and {\em high probability}. In a first phase, high-probability paths are explored in a way that maximizes path diversity, generating a first set of test inputs. In the second phase, low-probability paths are analyzed using symbolic execution, generating a second set of test inputs that should expose executions characterized by best-execution times and by worst-execution times. Finally, the program is executed using the test inputs generated during the two phases and running times are measured to generate performance distributions. 

	Another interesting application of symbolic execution is presented by~\cite{PPM-CSF18}. Their technique exploits model counting and symbolic execution for computing quantitative bounds on the amount of information that can be leaked by a program through side-channel attacks. 
	
}
%As it is based on {\sc Angr}, Driller adopts an index-based memory model as in Section~\ref{ss:index-based-memory} where reads can be symbolic and writes are always concretized. % read/write addresses

\subsection{Bug Exploitation}
\label{ss:bug-exploitation}
Bugs are a consequence of the nature of human factors in software development and are everywhere. Those that can be exploited by an attacker should normally be fixed first: systems for automatically and effectively identifying them are thus very valuable.

{\sc AEG}~\cite{AEG-NDSS11} employs preconditioned symbolic execution to analyze a potentially buggy program in source form and look for bugs amenable to stack smashing or return-into-libc exploits~\cite{PB-SSP04}, which are popular control hijack attack techniques. The tool augments path constraints with exploitability constraints and queries a constraint solver, generating a concrete exploit when the constraints are satisfiable. The authors devise the {\em buggy-path-first} and {\em loop-exhaustion} strategies (Table~\ref{tab:heuristics}) to prioritize paths in the search. On a suite of 14 Linux applications, {\sc AEG} discovered 16 vulnerabilities, 2 of which were previously unknown, and constructed control hijack exploits for them.

{\sc Mayhem}~\cite{MAYHEM-SP12} takes another step forward by presenting the first system for binary programs that is able identify end-to-end exploitable bugs. It adopts a hybrid execution model based on checkpoints and two components: a concrete executor that injects taint-analysis instrumentation in the code and a symbolic executor that takes over when a tainted branch or jump instruction is met. Exploitability constraints for symbolic instruction pointers and format strings are generated, targeting a wide range of exploits, e.g., SEH-based and jump-to-register ones. Three path selection heuristics help prioritizing paths that are most likely to contain vulnerabilities (e.g., those containing symbolic memory accesses or instruction pointers). A virtualization layer intercepts and emulates all the system calls to the host OS, while preconditioned symbolic execution can be used to reduce the size of the search space. Also, restricting symbolic execution to tainted basic blocks only gives very good speedups in this setting, as in the reported experiments more than $95\%$ of the processed instructions were not tainted. {\sc Mayhem} was able to find exploitable vulnerabilities in the 29 Linux and Windows applications considered in the evaluation, 2 of which were previously undocumented. Although the goal in {\sc Mayhem} is to reveal exploitable bugs, the generated simple exploits can be likely transformed in an automated fashion to work in the presence of classical OS defenses such as data execution prevention and address space layout randomization~\cite{Q-SEC11}. 

\vspace{-1mm} % TODO
\subsection{Authentication Bypass}
\label{ss:auth-bypass}
Software backdoors are a method of bypassing authentication in an algorithm, a software product, or even in a full computer system. Although sometimes these software flaws are injected by external attackers using subtle tricks such as compiler tampering~\cite{KRS-TR74}, there are reported cases of backdoors that have been surreptitiously installed by the hardware and/or software manufacturers~\cite{CZF-USEC14}, or even by governments~\cite{NSA-BACKDOOR}. 

Different works~\cite{DMR-USEC13,ZBF-NDSS14,FIRMALICE-NDSS15} have exploited symbolic execution for analyzing the behavior of binary firmwares. Indeed, an advantage of this technique is that it can be used even in environments, such as embedded systems, where the documentation and the source code that are publicly released by the manufacturer are typically very limited or none at all. For instance,~\cite{FIRMALICE-NDSS15} proposes Firmalice, a binary analysis framework based on {\sc Angr}~\cite{ANGR-SSP16} that can be effectively used for identifying authentication bypass flaws inside firmwares running on devices such as routers and printers. Given a user-provided description of a privileged operation in the device, Firmalice identifies a set of program points that, if executed, forces the privileged operation to be performed. The program slice that involves the privileged program points is then symbolically analyzed using {\sc Angr}. If any such point can be reached by the engine, a set of concrete inputs is generated using an SMT solver. These values can be then used to effectively bypass authentication inside the device. On three commercially available devices, Firmalice could detect vulnerabilities in two of them, and determine that a backdoor in the third firmware is not remotely exploitable.


% Bibliography
%\bibliographystyle{abstract} 
\bibliographystyle{apalike-refs}
\bibliography{symbolic}

% History dates
%\received{--- 2016}{--- XXXX}{---- XXXX}

\end{document}

% End of v2-acmsmall-sample.tex (March 2012) - Gerry Murray, ACM


