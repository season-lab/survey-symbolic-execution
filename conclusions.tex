% !TEX root = main.tex

\section{Conclusions}
\label{se:conclusions}

Techniques for symbolic execution have evolved significantly in the last decade, leading to major practical breakthroughs. In 2016, DARPA has challenged the global innovation community with a \$4M prize to build a computer that can hack and patch unknown software with no one at the keyboard. The winner, {\sc Mayhem}~\cite{MAYHEM-SP12}, was also the first autonomous computer system to play the Capture-The-Flag contest at the DEF CON 24 hacker convention\footnote{\url{https://www.defcon.org/html/defcon-24/dc-24-ctf.html}.}. The event demonstrated that tools for automatic exploit detection based on symbolic execution can be competitive with human experts, paving the road to unprecedented applications and the rise of start-ups that have the potential to shape software security and reliability the next decades. 

This survey has discussed some of the key aspects and challenges of symbolic execution, presenting them for a broad audience. For a number of related issues such as symbolic execution of multi-threaded programs~\cite{KPV-TACAS03,SA-HVC06,CLOUD9-EUROSYS11,FHR-ESEC13,BGC-OOPSLA14,GKW-ESEC15} and programs that manipulate floating point data~\cite{M-STVR01,BGM-STVR06,LTH-ICTSS10,CCK-EUROSYS11,BVL-POPL13,CCK-TSE14,RPW-SIGSOFT15}, which we considered beyond the scope of this article, we refer the interested readers to the cited body of works.


